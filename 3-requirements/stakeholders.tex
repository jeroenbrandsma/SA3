%!TEX root = ../report.tex
\section{Stakeholders and their concerns}

% wonder where do these factors come from

%kind of QA's (accourding to "Software Requirements" 3rd edition, Karll Wiegers and Joy Beatty)
%from page 263:

This section defines all stakeholders of our system and describe the concerns of the stakeholders. A stakeholder might be a person, group of persons, or organization that are involved in our system. There are ten stakeholders, ranged from first parties to third parties stakeholders. We use several quality standards from the "Software Requirements" book by Microsoft \cite{wiegers2013software}. Those quality standards are described in \autoref{table:qa_standard}.

\begin{table}[!htbp] \centering
	\caption{Quality attributes of Software Architecture from "Software Requirements" Book \cite{wiegers2013software}.}
	\label{table:qa_standard}
	\begin{tabular}{L{\tw{0.2}} L{\tw{0.4}}}
		\toprule
		\textbf{Quality Attributes} & \textbf{Brief description}                                                                                        \\ \midrule
		Availability                & The extent to which the system's services are available when and where they are needed                            \\
		Interoperability            & How easily the system can interconnect and exchange data with other systems or components                         \\
		Performance                 & How quickly and predictable the system responds to user inputs or other events                                    \\
		Reliability                 & How reliable the results of the system are (accuracy). \\
		Security                    & How well the system protects against unauthorized access to the application and its data                          \\
		Usability                   & How easy it is for people to learn, remember, and use the system                                                  \\
		\bottomrule
	\end{tabular}
\end{table}

There are six quality attributes, as can be seen in \autoref{table:qa_standard}, for measuring stakeholders' concern regarding our system. Furthermore, we also add profitability as another quality standard to improve measuring stakeholders' concern. Detailed description of stakeholders and their concerns are explained below.

% %from page 209 (Software Requirements book from microsoft):
% \todo{quote book, and ISO}
% Book and (ISO/IEC/IEEE 2011):\\
% \begin{tabular}{|L{\tw{0.2}}|L{\tw{0.4}}|}
% \toprule
% \textbf{Keyword} & \textbf{Priority} \\
% Shall & Required \\
% Should & Desired \\
% May & Optional \\
% \bottomrule
% \end{tabular}

\begin{description}
	\item[Product owner] is concerned about the reliability and profitability of the system. The product owner funds the whole project and is highly concerned about the profitability. Thus, to gain big market share and extract large profit from this product, the product owner has to make this product reliable.
			 
	\item[Developers] are concerned about interoperability, performance and security. We, the architect team of RugSAG3 company, are also part of this. This stakeholder is responsible for the development of the systems until it is ready for production. Including architecting, designing, analyzing, testing and implementing this SFM System.
	
	\item[Maintainers] are concerned about the reliability and availability of the system. These stakeholders are responsible for repairing sensors which break down and monitoring the system to act on reported faults.
			
	\item[Third party developers] are concerned about interoperability, availability, usability and reliability. Third party developers are important for our system since they need to provide an application that will give the users guidance in case of a flood.
			
	\item[Competitors] are concerned about reliability and profitability. Competitors give negative effect on the system because competitors will be aiming on the same customer target. On the other hand, competitors are also triggering us to make a really good system in order to be able to compete with them and to save more lives. Thus, competitors must also be kept in consideration.
			
	\item[Government] is concerned about availability and reliability. The government will be the main customer of this product, specifically, The Dutch Ministry of Infrastructure and the Environment. The government will be part of mitigation when the flood is imminent. This system will help the government by notifying them when it detects a flood and supplying them with relevant information about the flood.
			
	\item[Citizens] are also concerned about availability and reliability, but also usability, since they can subscribe to warning by SMS. The Dutch residents are indirect user of this systems. Furthermore, they want this system to always be available and run correctly and notify them with reliable information.
	% Guntur: I do not know the exact position of insurance companies in our stakeholders. Can anybody explain about this? Or should we just remove this stakeholder?
			
	\item[Insurance companies] are concerned mostly about performance, reliability and availability. The damages caused by flood sometimes are also covered by the insurance companies. Thus, the insurance companies will also be part of the stakeholders and they will make sure that their business is running well.
			
	\item[Local companies] are concerned about availability and reliability. Local companies will also be affected by the flood, since they have a lot of resources which are in danger. Local companies want to know whether or not this system is reliable so that they can arrange a proper action set when the flood comes to save their assets.
			
	\item[Safety region] is responsible for the emergency services and is concerned about interoperability, performance, availability, reliability and usability. Emergency services are important when any accident happens, including flood. They will be really concerned about the thing that makes this system reliable, and inter-operable to their current system.
\end{description}

\autoref{table:stakeholder_concern} illustrates the stakeholder concern matrix. In our approach every stakeholder is assigned a weight according to how important that stakeholder is. 

\begin{table}[!htbp] \centering
	\caption{Matrix of stakeholders concern.}
	\label{table:stakeholder_concern}
	\begin{tabular}{@{} cl*{11}c @{}}
		&  & \multicolumn{7}{c}{\textbf{Concerns}} \\[2ex]
		& \textbf{Stakeholder} & \rot{Weight} & \rot{Availability} & \rot{Interoperability} & \rot{Performance} 
		& \rot{Reliability} & \rot{Security} & \rot{Usability} & \rot{Profitability}\\
		\cmidrule[1pt]{2-10}		
		 %                	   weight ava  inte  perf  reli sec   usa  prof
		  & Product owner       & 1 &     &    &     & 60  &    &    & 40  \\
		  & Developers          & 1 &     & 40 & 30  &     & 30 &    &     \\
		  & Maintainers         & 1 & 60  &    &     & 40  &    &    &     \\
		  & Competitors         & 1 &     &    &     & 40  &    &    & 60  \\
		  & Government          & 2 & 40  &    & 20  & 40  &    &    &     \\
		  & Citizens            & 2 & 35  &    & 10  & 35  &    & 20 &     \\
		  & Insurance companies & 1 & 35  &    & 15  & 50  &    &    &     \\
		  & Local companies     & 1 & 60  &    &     & 40  &    &    &     \\
		  & Safety region       & 3 & 20  & 10 & 25  & 30  &    & 15 &     \\
		\cmidrule{2-10}
		  & Total               &   & 365 & 70 & 180 & 470 & 30 & 85 & 100 \\
		\cmidrule{2-10}
	\end{tabular}
\end{table}

As can be seen from \autoref{table:stakeholder_concern}, the most important concern of our system is the reliability, following availability as the second most important concern. This is also identical with our significant key driver.
