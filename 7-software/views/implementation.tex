\clearpage
\subsection{Implementation view}
\copied{The development view illustrates a system from a programmer's perspective and is concerned with software management. This view is also known as the implementation view. It uses the UML Component diagram to describe system components. UML Diagrams used to represent the development view include the Package diagram.}
{from wikipedia\\\url{https://en.wikipedia.org/wiki/4\%2B1_architectural_view_model}}

The implementation view, also known as the development view, describes the system from the programmer's perspective. This includes a description of the components of the system and how the system is packaged.  

\subsubsection{Components}
The system consists of a set of components tat all interact with each other. In figure~\ref{fig:component} below, these different components are shown. Each component can provide interfaces and uses sockets to connect to interfaces of other components. This way, the implementation of each component can independently be replaced and deployed. 

\clearpage
\begin{figure}[H]
	%\centering
	\includegraphics[keepaspectratio=true,height=13cm,width=0.9\textwidth]{{\viewimages/component}.jpg}
	\caption{Component diagram}
	\label{fig:component}
\end{figure}

There components are placed in three layers. The data source  layer, domain layer and service source layer.

In the data source layer the data is stored and the data mapper. It consist of two components. 
	Database: This is the database that holds all the stored data of the system.
	DataMapper: This component is responsible for reading from the database and adding new data.

In the domain layer the business rules and logic is placed. The following components are placed:
	UserData: Responsible for gathering the user data and normalizing
	SensorData: Responsible for gathering sensor data and normalizing
	FloodData: Responsible for gathering flood data and normalizing
	UAVData: Responsible for gathering UAV data and normalizing
	UserManager: Responsible for maintaining the phone numbers and other information of the registered users.
	DataProvider: Responsible for obtaining the data the REST server requests. 
	SensorManager: Component that is used for installing, update and delete sensors and retrieving data from the sensors.
	Algorithms: Component used for algorithms that calculated the flood probability.
	RESTController: Component to enable RESTful support.
	
In the Service layer the services are included used by external actors. The following components are included:
	RESTServer: Server which communicates with external software
	MessageDistribution: Component responsible for distributing warning messages to external parties.
	
Outside the layer are external software components:
	SMSService: Component to let citizen subscribe for warning texts and receive warning texts.
	SafetyRegion: Component to which the system communicates to in order to warn the Safety region.

\subsection{Database}
The figure \ref{fig:database} below shows how the database is structured. 

\clearpage
\begin{figure}[H]
	%\centering
	\includegraphics[height=14cm, width=0.9\textwidth]{{\viewimages/database}.jpg}
	\caption{Database diagram}
	\label{fig:database}
\end{figure}

The database diagram shows all the information the API can provide to the users. \\
The database is made of ten tables. \\

In green is the name of the table. \\
In grey the primarykey. \\
In the second column of each table are the types of the data. \\

The first table \textbf{aeracitizen} concerns the area where live the citizens : citizengeold (?) , area : the name of the area and citizencount gives the number of citizen in this area.\\
The table \textbf{aeraaltitude} gives information about the are such as its altitude.
The table \textbf{UAVdata} provides information sent by the UAVs which will give more information about the flood such as : uavid which identifies a specific uav, datetime, altitude , latlong ,gpssatelite,mpegimage, \\
The table \textbf{registereduser} concerns information to identify a specific user to provide him the warning, provide guidance and help the safety region : his id, phonenumer and if he is registered or not. \\

Six table are related and provide information about the sensors : \\
verification (?) \\
The table \textbf{Floodsensorverification} (?) \\
The table \textbf{flood}gives information about the flood : its id by floodid , starttime , endtimide , area. \\
The table \textbf{monitor} : monitorid , its location, attributename. \\
The table \textbf{sensor} : sensorid identifies each sensor, its location , which monitor it depends on,and its type \\
The table \textbf{sensordata} : sensordataid which identifies each data of the sensor, sensorid, datetime which gives the exact time of when the data has been sent ,value,  rawvalue(?). 

\subsubsection{Packages}
The software of the system is divided into several packages. These packages and their relations can be seen in figure~\ref{fig:package-diagram}. 
In this diagram, «use» depicts the use of an interface exposed by a software package, «access» depicts a private import of (parts of) another package, while «import» means a public import of (parts of) another package. 

The diagram contains software packages of the system itself, as well as software packages provided by third parties. The software packages provided by third parties are drawn outside of the `Smart Flood Monitoring System'-box.

The `Third Party API' is the software package which exposes a REST-API, which can be used by third parties to query data from the system.
The `External Data' is the software package which uses APIs from other systems to query data, like weather and geographic data.

\begin{figure}[H]
	%\centering
	\includegraphics[keepaspectratio=true,width=1.0\textwidth]{{\viewimages/packages}.png}
	\caption{Package diagram}
	\label{fig:package-diagram}
\end{figure}

The following packages are in the system.
	\begin{itemize}
	 \item Sensor Software: This software package is used for retrieving data from the sensors. It uses the sensor manager.
	 \item Sensor Manager: Responsible for installing sensors and configure the sensors.
	 \item Data Abstraction: Holds all data that is stored in the system.
	 \item Third Party API: Package that enables third party software too access data from the system database.
	 \item External Data: This package retrieves data from the GEO API, Weather API and Demographic API.
	 \item Monitoring \& Flood detection: Access the data from database and calculates if a flood is imminent. 
	 \item Warning: When a flood is imminent the warning package is used to send out a warning.
	\end{itemize}

The following external packages are used:
\begin{itemize}
	\item SMS Service: This package is used to let citizen subscribe to the warning services and send out warnings.
	\item Safety region API: When a flood is imminent a warning is send by the system to the safety region API package.
\end{itemize}
