%!TEX root = ../report.tex
\section{Verification}
The feasibility of the \ProjectName{} is verified in the following section.

\subsection{Availability}
To determine the availability of the system we determine the availability of the separate components. Let:\\
MTBF: Mean Time Between Failure\\
MTTR: Mean Time To Repair \\
The availability of the complete system is given as $x$\\
$x = {MTBF \over{MTBF + MTTR}}$

\begin{figure}[h]
\centering
\begin{tabular}{l l l l}
\textbf{Component} & \textbf{MTBF} & \textbf{MTTR} & \textbf{Availability} \\ 
\hline Internet connection & 3 years & 1 day & 0.99908 \\ 
\hline Storage & 5 years & 1 day & 0.99945 \\ 
\hline API & 3 years & 1 day & 0.99908 \\
\end{tabular} 
\label{table:risk-severity}
\end{figure}
The system availability can be calculated based on the table above. This is done by multiplying all seperate availabilities. The result of this is 0.9976. This meets the requirement of the system to be available 99.7\% of the time.

\subsection{Cost}
The external services are not all free. This means this is another expense. To use the openweathermap the system needs the service of \$470 a month. The back-up for the openweathermap is forecast.io, this gives 1000 calls per day free, after that each call costs \$0.0001. 
The geographic and demographic data is free in the Netherlands. In case of an expansion to other countries, other third party services may be looked into.