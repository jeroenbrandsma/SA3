%!TEX root = ../report.tex
\section{Stakeholders and their concerns}

% wonder where do these factors come from
%kind of QA's (accourding to "Software Requirements" 3rd edition, Karll Wiegers and Joy Beatty)
%(Ill create a bibtex and everything in a moment, so they can be referenced correctly)
%from page 263:
\begin{tabular}{|L{\tw{0.2}}|L{\tw{0.4}}|}
\toprule
\textbf{External quality} & \textbf{Brief description} \\ \midrule
Availability & The extent to which the system's services are available when and where they are needed\\
Installability & How easy it is to correctly install, uninstall, and reinstall the application \\
Integrity & The extent to which the system protects against data inaccuracy and loss\\
Interoperability & How easily the system can interconnect and exchange data with other systems or components\\
Performance & How quickly and predictable the system responds to user inputs or other events\\
Reliability & How long the system runs before experiencing a failure \\
Robustness & How well the system responds to unexpected operating conditions\\
Safety & How well the system protects against injury or damage\\
Security & How well the system protects against unauthorized access to the application and its data\\
Usability & How easy it is for people to learn, remember, and use the system\\
\midrule
\textbf{Internal quality} & \textbf{Brief description} \\ \midrule
Efficiency & How efficiently the system uses computer resources\\
Modifiability & How easy it is to maintain, change, enhance, and restructure the system\\
Portability & How easily the system can be made to work in other operating environments\\
Reusability & To what extent components can be used in other systems\\
Scalability & How easily the system can grow to handle more users, transactions, servers, or other extensions\\
Verifiability & How readily developers and testers can confirm that the software was implemented correctly\\
\bottomrule
\end{tabular}

%from page 209 (Software Requirements book from microsoft):
\todo{quote book, and ISO}
Book and (ISO/IEC/IEEE 2011):\\
\begin{tabular}{|L{\tw{0.2}}|L{\tw{0.4}}|}
\toprule
\textbf{Keyword} & \textbf{Priority} \\
Shall & Required \\
Should & Desired \\
May & Optional \\
\bottomrule
\end{tabular}

There are eight stakeholders who are involved in our system. The stakeholders are ranged from first parties to third parties stakeholders. Detailed description is written below.

\begin{description}
\item[Product owner] is concerned about reliability, profitability, affordability. Product owner funds the whole project. Product owner highly concerns about the profitability. Thus, to gain big market share and extract large profit from this product, product owner has to make this product reliable. Furthermore, to compete with other competitors in this area, affordability will also be another concern.
\item[Developers] are concerned about reliability, maintainability, and testability. We, the architect team of RugSAG3 company, are also part of this. This stakeholder is responsible for the development of the systems until its ready for production. Including architecting, designing, analysing, testing and implementing this Smart Flood Monitoring System.
\item[Competitors] are concerned about reliability, adaptability, profitability, and affordability. Competitors give negative effect on the system because competitors will be aiming on the same customer target. On the other hand, competitors are also triggering us to make a really good system in order to be able to compete with them and to save more lives. Thus, competitors must also be kept in consideration.
\item[Government] is concerned about reliability, adaptability, and affordability. Government will be the main customer of this product, specifically, The Dutch Ministry of Infrastructure and the Environment. Government will be part of mitigation when the flood is imminent. This system will help the government by notifying them when this system detects flood and what is the recommendations to do next along with some data regarding this system's findings.
\item[Citizens] are concerned about reliability and adaptability. The Dutch residents are indirect user of this systems. However, they are also able to directly subscribe to this service and thus they want this system to be adaptable to their current technological viewpoint. Furthermore, they want this system to run correctly and notify them with the reliable information.
% Guntur: I do not know the exact position of insurance companies in our stakeholders. Can anybody explain about this? Or should we just remove this stakeholder?
\item[Insurance companies] are concerned about reliability of this system. The damages caused by flood sometimes are also covered by the insurance companies. Thus, the insurance companies will also be part of the stakeholders and they will make sure that their business is running well.
\item[Local companies] are concerned about reliability and adaptability. Local companies will also be affected by the flood, they also have a lot of resources that are in danger. Local companies want to know whether or not this system is reliable so that they can arrange a proper action sets when the flood comes to save their assets. Moreover, local companies are also willing this system to be as adaptable as possible to their current technological viewpoint.
\item[Emergency services] are concerned about reliability, adaptability, and testability. Emergency services are important when any accident happens, including flood. They will be really concerned about the thing that makes this system reliable, adaptable to their current system, and could be tested in order to make sure things are running correctly.
\end{description}

\autoref{table:stakeholder_concern} illustrates the stakeholder concern matrix. In our approach every stakeholders are equally the same. Thus, each stakeholder receives 100 points in total that has to be distributed among all the concerns.

\begin{table}[!htbp] \centering
	\caption{Matrix of stakeholders concern.}
	\label{table:stakeholder_concern}
    \begin{tabular}{@{} cl*{10}c @{}}
        & & \multicolumn{6}{c}{\textbf{Concerns}} \\[2ex]
        & & \rot{Reliability} & \rot{Adaptability} & \rot{Profitability} 
        & \rot{Affordability} & \rot{Maintainability} & \rot{Testability}\\
        \cmidrule[1pt]{2-8}		
        					  % reli adap prof aff  main test
        & Product owner			& 30 &    & 40 & 30 &    &    \\
        & Developers			& 40 &    &    &    & 30 & 30 \\
        & Competitors 			& 25 & 25 & 25 & 25 &    &    \\
        & Government 			& 55 &    &    & 45 &    &    \\
        & Citizens				& 70 & 30 &    &    &    &    \\
        & Insurance companies	& 100&    &    &    &    &    \\
        & Local companies		& 75 & 25 &    &    &    &    \\
 \rot{\rlap{\textbf{~Stakeholder}}}
        & Emergency services	& 45 & 30 &    &    &    & 25 \\
        \cmidrule{2-8}
        & Total                	& 440& 110& 65 & 100& 30  & 55 \\
        \cmidrule{2-8}
    \end{tabular}
\end{table}

As can be seen from \autoref{table:stakeholder_concern}, the most important concern of our system is the reliability, following adaptability as the second most important concern. This is also identical with our significant key driver.
