%!TEX root = ../report.tex
\chapter{Analysis}
\label{ch:analysis}
This chapter describes the analysis of the system. It lists the assumptions that are made about the system and its environment.  Next, an overview of the high-level design decisions is given.

\section{Assumptions}
There are several assumptions made about the system and its environment:

\noindent{
	\begin{enumerate}
		\item The safety region / government has means to alert citizens in an area to evacuate.
		\item The safety region is informed about our system and will alert citizens if needed when our system alerts the safety region.
		\item People who are subscribed for flood warnings/guidance have a mobile phone that can receive text messages.
		\item Sensors can be placed in the water ways and dikes in locations where they can provide representative measurements.
		\item Information is available to the system with regards to the population density and terrain height in different areas.
		\item Supplying information about the areas affected by the flood in with a radius of 5 km (as described in \ref{fr:compute-area}) is sufficient for the government/safety region to make decisions regarding alerting and evacuating citizens.
		\item It is possible to predict the expected water level up to 12 hours in the future.
		\item A weather forecasting API is available that provides precipitation and wind data.
		\item The safety region provides an API, which can be used by the system to warn the safety region.
		\item The information about the flood warning can be displayed in the safety region using an API provided by them. 
	\end{enumerate}
}

\newpage
\section{High-level Design Decisions}
This section discusses the high-level design decisions.

Each decision is explained in a table. The arguments section of the table lists for each alternative a score for every important quality attribute.
A higher score means a more favorable result. For example, a high score for costs means a low cost for the system. Information about the various alternatives was searched on the internet. This information was used for grading how well the alternatives scored on the quality attributes.
\\[1.0cm]

%\noindent{
%	\begin{itemize}
%	\item Security? (What if someone tries to manipulate a sensor?)
%	\item One or multiple warning systems (failover)
%	\item Networking? Redundancy of cables (trunked?)
%	\item When to alarm? What to do if only 1 sensor warns about a flood?
%	\item Security decisions?
%	\item Operating system decision
%	\item Storage decision
%	\item Push or pull updates? (Pushing I’m hoping)
%	\item Programming language(s) for the system? Web-interface, sensor monitoring, weather forecasting system, push notification system
%	\item In what ways will the system detect a flood? Sensors, UAV’s, forecasting?
%	\item How to detect a flood based on the forecasting? Warn beforehand? Pay extra attention to the new upcoming flood by sending extra robots/uav's?
%	\end{itemize}
%}

% Template for decision table:
% We removed the status from the orignal entry, because we do not use this document to get confirmation on our decisions
% \begin{table}
% \begin{tabular}{L{0.2\textwidth} L{0.6\textwidth}}
%     \textbf{Name} 			& \textbf{Topic} \\ \toprule
%     \textbf{Decision} 		& \textbf{DEC-}\textbf{\nextNrRef{dn}} \\ \midrule \midrule
%     \textbf{Problem/Issue} 	& A problem description \\ \midrule
%     \textbf{Decision} 		& The method/product that has be chosen and why\\ \midrule
%     \textbf{Alternatives} 	& \textit{Other option}\\ 
%     							& Details about the other options\\\midrule
%    \textbf{Arguments} 		& A short discussion of the cons and pro's supported by the table below\\
%    						& 	\begin{tabular}{l|lllllll|l}
% 							& 		\rot{Reliability} & \rot{Resilience} & \rot{Performance} & \rot{Interopertability} & \rot{Security} & \rot{Scalability} & \rot{Cost} & \rot{\textbf{Score}} \\ \hline 
% 									weight 				& ? & ? & ? & ? & ? & ? & ? & ?\\ \hline
% 									Mobile broadband 	& ? & ? & ? & ? & ? & ? & ? & ?\\
% 									landline 			& ? & ? & ? & ? & ? & ? & ? & ?\\
% 									satalite 		 	& ? & ? & ? & ? & ? & ? & ? & ?\\
% 									direct lines 		& ? & ? & ? & ? & ? & ? & ? & ?\\
% 								\end{tabular} \\ \bottomrule
% \end{tabular}
% \caption{Decision -- Decision name.}
% \label{table:caption_alias}
% \end{table}

\begin{table}[H]
	\begin{tabular}{L{0.2\textwidth} L{0.6\textwidth}}
		\textbf{Name} 			& \textbf{Operating system} \\ \toprule
		\textbf{Decision} 		& \textbf{\req{dec}} \\ \midrule \midrule
		\textbf{Status} 		& \textbf{Approved} \\ \midrule
		\textbf{Problem/Issue} 	& The warning system software for the natural disasters need a platform to work on.  \\ \midrule
		\textbf{Decision} 		&  The warning system will use Linux as a platform. Based on Unix, Linux is a free platform that has proven itself and is used by many servers. It's open source meaning that everyone can check out how it works. Specifically CentOS will be used, this is an operation system based on the Linux platform. \\ \midrule
		\textbf{Alternatives} 	& \textit{Windows}\\
		& Operating system is a closed platform developed by one of the biggest tech companies who provide a big development environment with it.\\
		& \textit{OpenBSD}\\
		& A Unix-based system that is famous for it's proactive security and runs most of the Linux applications. However, some software packages aren't certified to run on OpenBSD, but are for Linux.\\
		\midrule
		\textbf{Arguments} 		& \\
		& 	\begin{tabular}{l|lllllll|l}
		        & \rot{Reliability} & \rot{Resilience} & \rot{Performance} & \rot{Interopertability} & \rot{Security} & \rot{Scalability} & \rot{Cost} & \rot{\textbf{Score}} \\ \hline
		% 							rel res perf int sec sca cost
		Weight  & 1                 & 1                & 1                 & 1                       & 1              & 1                 & 1          &                      \\ \hline
		Linux   & 4                 & 4                & 4                 & 4                       & 4              & 4                 & 5          & 29                   \\
		Windows & 3                 & 2                & 3                 & 2                       & 3              & 3                 & 1          & 17                   \\
		OpenBSD & 5                 & 4                & 4                 & 3                       & 5              & 4                 & 3          & 28                   \\
	\end{tabular} \\
	\\ \bottomrule
	\end{tabular}
	\caption{Decision -- Operating system}
	\label{table:os}
\end{table}
%\todo[inline]{We either need to adjust some weights or we need to switch to OpenBSD. Also, I might be biased.}
% WM: we could make cost of BSD a lower score, because developers have less experience with it and it is more expansive because of that
%   WM: I adjusted some weights, but they are now equal
\newpage

\begin{table}
	\begin{tabular}{L{0.2\textwidth} L{0.6\textwidth}}
		\textbf{Name} 			& \textbf{Connectivity of the sensors} \\ \toprule
		\textbf{Decision} 		& \textbf{\req{dec}} \\ \midrule \midrule
		\textbf{Status} 		& \textbf{Approved} \\ \midrule
		\textbf{Problem/Issue} 	& The sensors need to deliver their data to the system.   \\ \midrule
		\textbf{Decision} 		&  The sensors will be connected over landline / internet cable. The large amount of Arduinos that are to be connected, make the cable a more scalable option. \\ \midrule
		\textbf{Alternatives} 	& \textit{Landline/internet cable}\\ 
		& Connecting the sensors to the cabled telephone/internet network and use that network to communicate with the server.\\
		& \textit{Satellite}\\ 
		& Set up a satellite connection between the sensors and the system.\\
		& \textit{Direct lines}\\ 
		& Connection the sensors to the system by drawing a cable from all sensors to the system.\\
		\midrule
		\textbf{Arguments} 		& \\
		& 	\begin{tabular}{l|llllllll|l}
		                 & \rot{Reliability} & \rot{Resilience} & \rot{Performance} & \rot{Interoperability} & \rot{Security} & \rot{Scalability} & \rot{Cost} & \rot{\textbf{Score}} \\ \hline 
		Weight           & 2                 & 2                & 1                 & 3                       & 1              & 2                 & 1          &                      \\ \hline
		Mobile broadband & 4                 & 4                & 2                 & 2                       & 3              & 2                & 2          & 33                   \\
		Landline/internet cable  & 3                 & 3                & 3                 & 4                       & 3              & 4                 & 5          & 43                   \\
		Satellite        & 3                 & 1                & 2                 & 4                       & 4              & 4                 & 1          & 35                   \\
		Direct lines     & 2                 & 1                & 5                 & 2                       & 5              & 1                 & 1          & 25                   \\
	\end{tabular} \\ \bottomrule
	\end{tabular}
	\caption{Decision -- Connectivity of the sensors}
	\label{table:connectivitysensors}
\end{table}

\newpage
\begin{table}
	\begin{tabular}{L{0.2\textwidth} L{0.6\textwidth}}
		% http://www.stevenswater.com/water_level_sensors/
		\textbf{Name} 			& \textbf{Type of water level sensor} \\ \toprule
		\textbf{Decision} 		& \textbf{\req{dec}} \\ \midrule \midrule
		\textbf{Status} 		& \textbf{Approved} \\ \midrule
		\textbf{Problem/Issue} 	& To measure the water level in the water ways and along the coast, a sensor is needed to measure the water level. \\ \midrule
		\textbf{Decision} 		& The system will use pressure sensors to measure the water level. Pressure sensors are submerged at a fixed level in the water body. By measuring the pressure on the sensor of the water above it, these types of sensors are able to determine the water level. \\ \midrule
		\textbf{Alternatives} 	
		&\textit{Float-operated sensor}\\ 
		& These types of sensors are mechanical and have a floating element which can move up and down with the water level. The floating element is protected in a `stilling well' and therefore, the risk of damage is low.  \\
		&\textit{Non-contact sensor}\\
		& These types of sensors use (ultra)sonic waves to determine the water level. Sediment in the water can cause issues with the measurements.\\
		&\textit{Bubbler sensors}\\
		& Bubbler sensors can measure the water level by measuring the pressure needed to force an air bubble through a submerged tube. Bubbler sensors have good resistance against damage from floods and debris. \\
		% http://www.fondriest.com/news/flowlevelmeasurements.htm
		\textbf{Arguments} 		& For this decision the interoperability and security have a weight of 0, because those depend on the specific sensor used and not on the type of sensor. Scalability was not taken into account because this depends mostly on the costs of the type of sensor.\\
		& 	\begin{tabular}{l|lllllll|l}
		                      & \rot{Reliability} & \rot{Resilience} & \rot{Performance} & \rot{Interopertability} & \rot{Security} & \rot{Scalability} & \rot{Cost} & \rot{\textbf{Score}} \\ \hline 
		% 								rel res perf int sec sca cost
		Weight                & 3                 & 2                & 1                 & 0                       & 0              & 0                 & 3          &                      \\ \hline
		Pressure sensor       & 3                 & 3                & 3                 & -                       & -              & -                 & 4          & 30                   \\
		% wide price range: 250 - 1500 usd
		Float-operated sensor & 3                 & 4                & 3                 & -                       & -              & -                 & 3          & 29                   \\
		% price ranges 450 - 800 usd
		Non-contact sensor    & 2                 & 4                & 3                 & -                       & -              & -                 & 2          & 23                   \\
		% because it is above water, more things affect measurements like air temp.
		% reliability is less because sediment in water can cause issues
		% price ranges 600 - 1000 usd
		Bubbler sensors       & 4                 & 5                & 2                 & -                       & -              & -                 & 1          & 29                   \\
		% response time 30s
		% could not find price, only on request, so probably expensive
	\end{tabular} \\ \bottomrule
	\end{tabular}
	\caption{Decision -- Type of water level sensors}
	\label{table:waterlevelsensortypeDecision}
\end{table}

\begin{table}
	\begin{tabular}{L{0.2\textwidth} L{0.6\textwidth}}
		\textbf{Name} 			& \textbf{Server} \\ \toprule
		\textbf{Decision} 		& \textbf{\req{dec}} \\ \midrule \midrule
		\textbf{Status} 		& \textbf{Approved} \\ \midrule
		\textbf{Problem/Issue} & To connect all sensors and to process all the data, we need computing power and storage space. \\ \midrule
		\textbf{Decision} 		& We will use our own server park. When managing our own server park, we have more technical freedom. Especially when our system will scale up, it can be profitable to maintain our own server park instead of a cloud solution.  \\ \midrule
		\textbf{Alternatives} 	
		&\textit{Cloud computing}\\ 
		& There are different cloud computing providers in the world. An advantage of cloud computing is the degree of scalability. However, when storing data at a cloud provider, we can never be sure if we are the only ones with access to our data.\\
		\textbf{Arguments} 		& For this decision we will not take resilience into account. Resilience can also be seen as reliability and interoperability. \\
		& 	\begin{tabular}{l|lllllll|l}
		                & \rot{Reliability} & \rot{Resilience} & \rot{Performance} & \rot{Interopertability} & \rot{Security} & \rot{Scalability} & \rot{Cost} & \rot{\textbf{Score}} \\ \hline 
		% 								rel res perf int sec sca cost
		Weight          & 3                 & -                & 2                 & 2                       & 3              & 2                 & 2          &                      \\ \hline
		Cloud computing & 4                 & -                & 4                 & 3                       & 3              & 5                 & 3          & 51                   \\
		Own server park & 3                 & -                & 4                 & 5                       & 4              & 3                 & 4          & 53                   \\
	\end{tabular} \\ \bottomrule
	\end{tabular}
	\caption{Decision -- Server park}
	\label{table:waterlevelsensortype}
\end{table}