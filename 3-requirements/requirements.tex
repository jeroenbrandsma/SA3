%!TEX root = ../report.tex
\chapter{Requirements}
\label{ch:requirements}
This chapter will describe the vision and use it to derive stakeholders to be able to properly write use cases and stories. These will be used to extract functional, commercial, technical and evolution requirements. Afterwards, a risk assessment will take place, to ensure that the project is not at great risk.
%!TEX root = ../report.tex
\section{Architectural vision}
[This needs to be edited, depending on our final vision of the project]\\
The smart flood monitoring system consists of multiple parts. First of all there is the monitoring part. This part monitors the current state of the environment. To achieve this, we need a lot of data. This data is obtained by sensors and weather data. First of all we use sensors to get the current water level of waterways. The data that is obtained through these sensors is monitored in a control station. In the control station works a person who keeps track of all the relevant information. 

In case of an imminent flood a warning will be issued to the government and the citizens who live in the threatened area. This can be done in multiple ways. We issue a warning to the government, who then in their turn, warns the citizens. They do this by using the sirens that are installed through the netherlands. Besides this they also use the NL-Alert service. We can also do this by warning the citizens ourselves. We do this by sending a text message to all the people that are in the threatened area. This is done by sending text messages to all the phones that are connected to the celltowers of the providers. 

After a warning is send, the citizens can use our smartphone application to get a proper guidance to a safe place. The application gives the shortest route and takes obstacles into account. A citizen can report an obstacle to our system. This means the guidance part of out system becomes flexible. 

%!TEX root = ../report.tex
\section{Stakeholder and their concerns}

% Key drivers
% high level requirements

%!TEX root = ../report.tex

%Citizen subscribes to service
%Sensor data is retrieved FR-1, FR-3, FR-5
%Weather data is retrieved FR-7
%Predict floud probability FR-8
%Get geo data FR-9
%The system should predict water levels FR-11, FR-12
%Monitor people should have access to a control panel FR-21

\clearpage


\section{Stories and use-cases}
This section will give an overview of the different use-cases. Figure \ref{fig:usecase-diagram} displays the use-case diagram. This provides an overview of the use-cases with their actors. In the subsections below, the architectural important use-cases are explained in more detail.

\begin{figure}[h]
\centering
\includegraphics[width=90mm]{images/usecaseDiagram.png}
\caption{Use-case diagram}
\label{fig:usecase-diagram}
\end{figure}


%slides say:
%	Name & \\
%	Number & \\
%	Primary actor &  \\
%	Scope &  \\
%	Level &  \\
%	Extensions &  \\
%	Sub-variations & \\


\subsection{Retrieve sensor data}
\pgfplotstabletypeset[%
UCTable
]{%
	value & description \\
	Number & \req{uc}\\
	Description & The system receives data from the different sensors deployed \\
	Stakeholders and interests & \compactList{itemize}{%
			\item \textbf{Developers}: Developers need to work with the sensor data
	}\\
	Primary actor & System\\
	Scope & Monitoring part of the system \\
	Level & Sub process\\
	Precondition & The sensor is connected to a processing unit \\
	Main success scenario & \compactList{enumerate}{%
			\item The sensor does a measurement
			\item The sensor sends the data to a processing unit
			\item The processing unit normalizes the received data
			\item The processing unit sends the normalized data to the main processing node
			\item The main processing node stores the data in the database
	}\\
	Postcondition & The system received and stored the sensor data \\
	Alternatives & \compactList{itemize}{%
			\item[2a.] The data can't be sent. \\
			Data will be lost \\
			The use-case ends 
	}\\
	Related requirements & FR-1, FR-2, FR-3, FR-4, FR-5, FR-6\\
}

\clearpage

\subsection{Retrieve weather data}
\pgfplotstabletypeset[%
UCTable
]{%
	value & description \\
	Number & \req{uc}\\
	Description & The system receives data from the weather forecast service \\
	Stakeholders and interests & \compactList{itemize}{%
			\item \textbf{Developers}: Developers would like to have a simple to use API
	}\\
	Primary actor & System\\
	Scope & Monitoring part of the system \\
	Level & Sub process\\
	Precondition & The system needs external weather data to predict floods \\
	Main success scenario & \compactList{enumerate}{%
			\item The processing unit determines it needs forecast weather data%
			\item A call is made to the weather forecast service%
			\item The weather forecast service returns the requested data
	}\\
	Postcondition & The system received the forecast data \\
	Alternatives & \compactList{itemize}{%
			\item[3a.] The data can't be returned. \\
			Repeat this process with another weather forecast service. \\
			If none are available, proceed monitoring without weather forecast data. \\
			After 5 minutes try to reconnect.
	}\\
	Related requirements & FR-7\\
}

\clearpage

\subsection{Citizens subscribe to the SMS service}
\pgfplotstabletypeset[%
UCTable
]{%
	value & description \\
	Number & \req{uc}\\
	Description & Citizens can subscribe to the SMS service, so when a flood happens they will get a direct text message\\	
	Stakeholders and interests & \compactList{itemize}{
			\item \textbf{Citizens}: Citizens want to be warned as soon as possible.
	}\\
	Primary actor & Citizen\\
	Scope & Warning part of the system \\
	Level & Sub process\\
	Precondition & Citizen has a mobile phone and is not subscribed to the SMS service \\
	Main success scenario & \compactList{enumerate}{
			\item Citizen sends a text message to our service
			\item The SMS service receives the text message
			\item The SMS service stores the phone number in the database
			\item A text message is sent back to the citizen with confirmation
	}\\
	Postcondition & Citizen is subscribed to the SMS service \\
	Alternatives & \compactList{itemize}{%
			\item[2a.] The text message is not received \\
				The use-case ends
	}\\
	Related requirements & FR-17\\
}

\clearpage

\subsection{Determining flood probability}
\pgfplotstabletypeset[%
UCTable
]{%
	value & description \\
	Number & \req{uc}\\
	Description & The central processing unit calculated the probability of a flood \\
	Stakeholders and interests & \compactList{itemize}{%
			\item \textbf{Developers}: Developers have to work on this part
			\item \textbf{Emergency Services}: Emergency services want to know when a flood warning is triggered
			\item \textbf{Government}: The government would also like to know when a flood warning is triggered
	}\\
	Primary actor & System\\
	Scope & Monitoring and warning part of the system\\
	Level & Main process\\
	Precondition & The sensor data is available \\
	Main success scenario & \compactList{enumerate}{%
			\item The central processing unit gets the latest sensor data from the database
			\item The central processing unit gets the latest weather forecast data
			\item The central processing unit calculates the probabilty of a flood
			\item The central processing unit stores the probability value in the database
			\item The central processing unit determines that a flood is imminent based on the probability value
			\item A warning is send to the emergency services
			\item A warning is send to the government
			\item A warning is send to the citizens
	}\\
	Postcondition & The flood probability is calculated and stored. If the probability exceeds a certain threshold, a warning is sent to the authorities and citizens \\
	Alternatives & \compactList{itemize}{
			\item[5a.] The probability is not above the threshold \\
			The use-case ends
	}\\
	Related requirements & FR-8\\
}

\clearpage

\subsection{Warn citizens in case of an imminent flood}
\pgfplotstabletypeset[%
UCTable
]{%
	value & description \\
	Number & \req{uc}\\
	Description & Citizens who are subscribed to the SMS service will be warned through text messages in case of an imminent flood\\	
	Stakeholders and interests & \compactList{itemize}{
			\item \textbf{Citizens}: When they are subscribed, they want to warned in case of an imminent flood
	}\\
	Primary actor & Citizen\\
	Scope & Warning part of the system \\
	Level & Sub process\\
	Precondition & There is an imminent flood and citizen is subscribed to the SMS service\\
	Main success scenario & \compactList{enumerate}{%
			\item The processing unit sends a warning about an imminent flood to the SMS service 
			\item The SMS service composes a list with phone numbers to warn
			\item The SMS service sends a warning to all phone numbers on the list
	}\\
	Postcondition & The citizens who are subscribed received a warning\\
	Alternatives & \compactList{itemize}{%
			\item[3a.] A message can't be sent to the citizen \\
				Wait a minute and resend\\
				The use-case ends
	}\\
	Related requirements & FR-18, FR-19\\
}

\clearpage

\subsection{Warn authorities in case of an imminent flood}
\pgfplotstabletypeset[%
UCTable
]{%
	value & description \\
	Number & \req{uc}\\
	Description & Government and emergency services receive a warning about an imminent flood\\
	Stakeholders and interests & \compactList{itemize}{%
			\item \textbf{Government}: The government wants to warn the citizens in case of a flood
			\item \textbf{Emergency services}: The emergency services want to help the citizens in case of a flood
	}\\
	Primary actor & Government, Emergency services\\
	Scope & Warning part of the system \\
	Level & Sub process\\
	Precondition & There is an imminent flood\\
	Main success scenario & \compactList{enumerate}{
			\item The processing unit determines what area will be under water in case of a flood
			\item The processing unit determines how many people will be affected by the imminent flood
			\item The processing unit predicts how the flood will develop in the following period 
			\item The processing unit will create a map based on the current state and predictions
			\item The processing unit sends the map to the government and emergency services
	}\\
	Postcondition & A map with current and predicted data is sent to the government and emergency authorities \\
	Related requirements & FR-10, FR-11, FR-12, FR-13, FR-14 \\
}

% %\todo{joris: "The sensors send their data to the processing unit" is removed here, but is the first mss item in the preview usecases}
% \pgfplotstabletypeset[%
% UCTable
% ]{%
% 	value & description \\
% 	Number & \req{uc}\\
% 	Name & Guide citizens\\
% 	Description & A citizen requests guidance to get to a safe place in a flooded area \\
% 	Stakeholders and interests & \compactList{itemize}{%
% 			\item \textbf{Citizens}: Citizens want to be guided to a safe place in case of a flood
% 			\item \textbf{Emergency services}: The emergency services want to help the citizens in times of need}\\
% 	Primary actor & Citizen\\
% 	Scope & Guidance part of the system \\
% 	Level & Main process\\
% 	Precondition & There is a flood and the citizen is subscribed to the warning service. \\
% 	Main success scenario & \compactList{enumerate}{%
% 			\item The processing unit determined there is a flood%
% 			\item The processing unit determines generic routes
% 			\item The processing unit turns this routes into a MMS message with the right directions
% 			\item The generic routes are send through MMS message to all subscribed phone numbers}\\
% 	Postcondition & Citizen received his/her personal route to safety \\
% 	Extensions & \compactList{itemize}{%
% 			\item[4a.] MMS message can't be send. Wait a minute and try to resend}\\
% 	%Sub-variations & \\
% }

%If we decide to use user input
% \subsection{A citizen reports an obstruction}
% \textbf{Scope:} Monitoring part of the system\\\\
% \textbf{Level:} Main process\\\\
% \textbf{Primary actor:} Citizen\\\\
% \textbf{Stakeholders and interests:}\\
% 	1. Citizen - A citizen wants to report an obstruction to make the guidance part more reliable. \\\\
% \textbf{Preconditions:} There is an obstruction which is not yet reported. \\
% \textbf{Postconditions:} There is an obstruction which is reported by a citizen and is known in the system. \\\\
% \textbf{Main succes scenario:} \\
% 1 - There is an obstruction somewhere in the area. \\
% 2 - The citizen opens our application on his smartphone. \\
% 3 - The citizen reports the kind of obstruction and the location. \\
% 4 - The system gets the obstruction information as input. \\
% 5 - The obstruction is processed in the system and visible to other citizens. \\\\
% \textbf{Extensions:} \\
% 3a - The citizen doesn't know the location, GPS can be used in this case. If GPS doesn't work, the obstruction can't be reported. \\
% 4a - The smartphone can't send the information. In this case, the obstruciton can't be reported. 


%!TEX root = ../report.tex
\section{Functional requirements}
\begin{longtable}{L{0.1\textwidth} L{0.12\textwidth} L{0.78\textwidth}}


    \textbf{Nr.} & \textbf{Prio}  & \textbf{Description} \\
    
    \frReqRow{1}{Must}{The system is able to receive input from sensors with regards to the water level. This information will be used to determine if there is an imminent flood.}
    % GK: In what way is the information delivered from the sensors?
    %   WM: Is that not for analysis?
    
    \frReqRow{2}{Must}{ The system is able to process input from sensors with regards to the water level. }
    
	\frReqRow{3}{Must}{The system is able to receive input from sensors with regards to the pressure/consistency of the dykes.}
	
	\frReqRow{4}{Must}{The system is able to process input from sensors with regards to the pressure/consistency of the dykes. This information will be used to determine if there is an imminent flood.}

    \frReqRow{5}{Must}{The system retrieves weather forecasting data from weather forecasting services. The retrieved weather forecasting data consists of predictions about the precipitation and wind data and is used by the system to help in determining when a flood becomes imminent.}
     
    \frReqRow{6}{Must}{ The system is able to detect when a flood is imminent by using the retrieved sensor data and weather forecasting data. }
    
    \frReqRow{7}{Must}{ The system has access to geographic information, including road data and terrain height data.}
        
    \frReqRow{8}{Must}{ The system can determine the area affected by a flood, by using the location data of the sensors and geographic information.}
    
    \frReqRow{9}{Must}{ The system computes (from geographic information) the area which will be affected by a flood.}
    %GK: be careful with exactness in this, add some estimation

    \frReqRow{10}{Must}{ The system collects information pertaining to the severity of the flood: the expected water level, how fast the water level in the flood area will rise, and the number of civilians living in the affected area (population density).}
    
    \frReqRow{11}{Must}{ The system provides emergency services with information about the flood. This includes the area affected by the flood and information needed to deduct the severity of the flood.}
    
	\frReqRow{12}{Must}{ When a flood is imminent, the system sends a warning to the emergency services and to the authorities. The warning contains information about the area affected by the flood. }
    % what info is in the warning
    % who is that: authorities?
    
    \frReqRow{13}{Must}{ The system is able to compute a safe route, free of obstruction and floods, to a safe area, not affected by the flood, where citizens can be evacuated to in case of an (imminent) flood.}
    
    \frReqRow{14}{Must}{The system is able to receive and process input from sensors with regards to the pressure/consistency of the dykes. This information will be used to determine if there is an imminent flood.}
    
    \frReqRow{15}{Must}{Citizens who are subscribed for flood warnings are warned about imminent floods through by text message.}
    
    \frReqRow{16}{Must}{The system can determine the location of citizens who are subscribed for flood warnings.}
    
    \frReqRow{17}{Must}{Citizens who are subscribed for flood warnings get route information to a safe area by MMS.}
    
    \frReqRow{18}{Must}{The location of citizens who are warned through the app can be determined in order to supply them with a safe route to a safe area.}
	
    \frReqRow{19}{Must}{ The system is able to predict the development of the water level. This information can be used to predict how fast a flood will develop. }
	
    \frReqRow{20}{Must}{ The system uses different sources to confirm imminent flood warnings, in order to limit false positives. } % not very specific yet
	
    \frReqRow{21}{Must}{ The system can detect a faulty sensor, either when the sensor raises an error or when the data from the sensor is inconsistent with other sensor data. }
    
    \frReqRow{22}{Must}{ There is a control panel, where an overview of warnings and errors of the system can be viewed. }
	
    \frReqRow{23}{Must}{ The system can detect and report faulty sensors, so these sensors can be repaired or replaced. } %how will it be reported and to whom?? \\
    
	\frReqRow{24}{Must}{ In case of a flood, the system provides emergency services with safe routes to incident locations. }
    
    \frReqRow{25}{Must}{ The system can determine the location of an emergency vehicle, so it is able to compute a safe route to incident locations. }

    \frReqRow{26}{Optional}{ The system is able to detect extreme weather phenomena, like storms etc. }
    
    \bottomrule
\end{longtable}

%!TEX root = ../report.tex
\section{Commercial non functional requirements}
 In this section commercial non functional requirements are presented.
 
\textbf{CNFR-1} The system is affordable.
The selling price of the system to the authorities is about *********** euros per city or municipality. This price is lower than 70\%\ of the competitors price. % How to calcuatenthe selling price , do we need to give it ?

%\textbf{CNFR-2} The maintenance costs are ********** euros.

\textbf{CNFR-2} The sensors have a good quality , the sensors companies have good ratings so we don't have replace the sensors often => less money spent on repairs. The guarantee of the sensors should be about three years.

IDEAS :
\textbf{CNFR-} A video explains how the system works to the end-users : authorities and emergency services.
\textbf{CNFR-} Advertissement ???

%!TEX root = ../report.tex

\section{Technical non-functional requirements}
This section describes the technical aspects that are important to the system as requirements. These requirements determine various APIs and programs that the system will rely on.

% In this section, the technical non-functional requirements important to this system are discussed.

\subsection{Reliability}
% The response time should be about NUMBER mS
% Failure rate ?
% Data transmission
% The margin of error tolerated has to be as low as possible .\\ GK: Margin of error in what aspect?
Reliability is an important non-functional requirement for the system, and a key-driver of the architecture as well.
\begin{longtable}{L{0.1\textwidth} L{0.12\textwidth} L{0.7\textwidth}}
	\textbf{Nr.} & \textbf{Prio}  & \textbf{Description} \\
	%\reqRow{rel}{1}{Must}{Sensor sites are equipped with at least two sensors} ( why two ? )
	\reqRow{rel}{1}{Must}{Data from the sensors is sent via a TCP connection} % Already mention REST here?
	 
	\reqRow{rel}{2}{Must}{The system must detect if a sensor supplies wrong measurements, which can be caused, e.g. by improper calibration or defects in the sensor.} 
	
	\reqRow{rel}{3}{Must}{The system must at no time fail to detect a flood when this flood becomes imminent (\textit{false negative}).}
	
	\reqRow{rel}{4}{Must}{The system must not detect a flood, when this flood is not there in reality (\textit{false positive}), on average more than once per 5 years.}
	 
	%\reqRow{rel}{3}{Must}{Redundancy} % GK: if redundancy is required, which it should be, in what way?
	%This means adding redundancy to the system so that failure of a component does not mean failure of the entire system
	\bottomrule
\end{longtable}

\subsection{Availability}
\begin{longtable}{L{0.1\textwidth} L{0.12\textwidth} L{0.7\textwidth}}
	\textbf{Nr.} & \textbf{Prio}  & \textbf{Description} \\
	\reqRow{ava}{1}{Must}{The system must have an uptime of $99.7\%$. This effectively means, that the system should not be down for more than 2 hours per month. $AV = \frac{\text{MTTF}}{\text{MTTF} + \text{MTTR}} = \frac{\text{6 months}}{\text{6 months }+\text{ 12 hours}} = \frac{4380 \text{ hours}}{4380 + 12 \text{ hours}} = 99.7 \%$}

	\reqRow{ava}{2}{Must}{The system must not experience a period of downtime, spanning more than 12 hours. Within twelve hours of the system going offline, it should be back up again. }
	\bottomrule
\end{longtable}


\subsection{Resilience}
The system needs to be resilient to recover from errors and mistakes without impacting the systems functionality.
\begin{longtable}{L{0.1\textwidth} L{0.12\textwidth} L{0.7\textwidth}}
	\textbf{Nr.} & \textbf{Prio}  & \textbf{Description} \\
	
	\reqRow{res}{1}{Must}{The system recognizes failures within half an hour}

	
	\reqRow{res}{2}{Must}{The system recovers from failures without the \qos or the functionality of the system being affected.} % GK: split in two requirements. reduce to detect sensor failures within half an hour
	
	\reqRow{res}{3}{Must}{All system data must be backed up every 24 hours, so that in case of data loss, this data can be restored.}
	
	\reqRow{res}{4}{Must}{In case of a data loss, the data should be retrieved and restored from a backup within 2 hours.}
	
	\reqRow{res}{5}{Must}{Backup copies are stored in a secure location which is not in the same area as the system (50 km).}
	
	\bottomrule
\end{longtable}

\subsection{Performance}

\begin{longtable}{L{0.1\textwidth} L{0.12\textwidth} L{0.7\textwidth}}
	\textbf{Nr.} & \textbf{Prio}  & \textbf{Description} \\
	\reqRow{perf}{1}{Must}{Data is transmitted from and to the system with a minimum average speed of 10 megabits per second} % How to calculate GK: Doesn't really matter for requirements, but good to think about. Also: response time to what? Request? Detecting floods? Furthermore, 10 (just saying something) GB of data is not transferable witihin 1 second if you include connection setup time etc etc
	% GK: only about datatransmission express in dataspeed
	
	\reqRow{perf}{2}{Must}{	The data transmission between the sensors and the system is on average at least 10 megabits per second for each sensor.}
	% How many data the system can receive in one second ?
	\reqRow{perf}{3}{Must}{	The time for the system to compute if there is a flood or not according to a critical level and the data received from the sensors is at most 5 minutes. }
	
	\reqRow{perf}{4}{Must}{ If an imminent flood is detected, the warning text message to citizens arrives in 5 minutes. }
	
	\reqRow{perf}{5}{Must}{ If an imminent flood is detected, the warning to the emergency room arrives within 1 minute. }

	\bottomrule
\end{longtable}
% GK: requirement for amount of concurrent users

%\includegraphics[scale=0.7]{3-requirements/Images/Performance.jpg}

\subsection{Interoperability}
The system has dependencies on several third-party systems and also allows third parties to retrieve information from it.

\begin{longtable}{L{0.1\textwidth} L{0.12\textwidth} L{0.7\textwidth}}
	\textbf{Nr.} & \textbf{Prio}  & \textbf{Description} \\
	\reqRow{intr}{1}{Must}{The system pulls weather forecasts from at least two weather forecasting services.}
	%\reqRow{intr}{3}{Must}{When the system detects a high risk of flood, it warns the government automatically using the government provided API.} WM: removed this, the safety region will warn the government (mayor etc.)
	\reqRow{intr}{2}{Must}{When the system detects a flood, it notifies the safety region through an API provided by them.}
	\reqRow{intr}{3}{Must}{The system sends out a SMS to all users who are subscribed to flood warnings using the mobile network.}
	\reqRow{intr}{4}{Must}{The systems is able to connect to different types of sensors.}
	\reqRow{intr}{5}{Must}{The system is able to retrieve geographical data from an API.}
	\reqRow{intr}{6}{Must}{The system exposes an API, allowing third parties to develop applications using the systems data.}
	\reqRow{intr}{7}{Must}{The sensors are able to communicate with the system using a mobile broadband connection.}

	\bottomrule
\end{longtable}
\todo[inline]{intr-6 Not measurable}
% GK; Requirement about working with different kind of sensors for the same data

\subsection{Security}
The security of the system is very relevant to its success. The system should be secure, because unauthorized access can have a big impact on society (when e.g. false flood warnings are triggered).
\begin{longtable}{L{0.1\textwidth} L{0.12\textwidth} L{0.7\textwidth}}
	\textbf{Nr.} & \textbf{Prio}  & \textbf{Description} \\
	\reqRow{sec}{1}{Must}{Access to the system is restricted to users, which are authorized and authenticated using a password protected user account.}
	\reqRow{sec}{2}{Must}{All communication to, from and within the system are encrypted.}
	\reqRow{sec}{3}{Must}{User account information is hashed using bcrypt after being salted with 128 randomly generated characters.}
	\reqRow{sec}{4}{Must}{The system is protected by a firewall that at least scans at the application layer, while also scanning for and mitigating DDoS attacks.}
	% WM: does communication within the system mean between the different components of the system?
	\reqRow{sec}{5}{Must}{The system communicates with its sensors via a REST API that only allows for HTTPS connection.} % Do we want to communicate with the sensors via REST? WM: Could this be more generic? Like: 'the system communicates with the sensors via a secure connection'

	\bottomrule
\end{longtable}
% TODO: how to measure?

\subsection{Scalability}
The system has to be designed in a way that it can expand over time. Not only should it span larger geographic areas, but more functionality will be added later as well.
% The system will be able to cover more areas and use more sensors in the future. GK: too vague, but added scale-3 based on this idea
\begin{longtable}{L{0.1\textwidth} L{0.12\textwidth} L{0.7\textwidth}}
	\textbf{Nr.} & \textbf{Prio}  & \textbf{Description} \\
	\reqRow{scale}{1}{Must}{ The database and services of the system should run in a cloud environment where they can scale, when the systems resource usage increases. }

	\reqRow{scale}{2}{Must}{The system is configurable to run in different areas and with different sensors.}
	
	\reqRow{scale}{3}{Must}{ The system maintains the performance requirements when the geographic area the system covers is expanded. }
	
	\bottomrule
\end{longtable}
% The system will be developed with more sensors and in more places.
%New functionalities, for example GPS
% ============================================================
% From Gerrit:
% The three main requirements:
% 1: Monitoring activities: the system should monitor activities and properties of
% rivers, waterways, dykes, such as the water level and pressure or the consistency
% of the dyke. Monitoring can be performed through different devices, e.g. analog
% and digital sensors, Unmanned Aerial Vehicles (UAVs) and Vehicular Ad-hoc
% Networks (VANETs) etc.
% 2: Warning: in case of an imminent flood, the system should issue warnings to the
% authorities and emergency services, but also directly to citizens who are
% subscribed for such messages (e.g. through SMS or mobile apps).
% 3: Guidance: the system should provide runtime information to guide its
% constituents or third parties, .e.g. the UAVs may provide information to
% VANETs embedded in vehicles crossing the area, thereby guiding vehicles
% driving towards a flood area to avoid certain routes. Meanwhile, the VANETs
% can also provide an alternative route to emergency services or citizens to avoid
% the flood area. 
% ToDo:
% Stakeholders
% Use-cases
% Make them smart
% Assign value (must/should/...
% ============================================================

% %!TEX root = ../report.tex
\section{Evolution requirements}
% GK: Im all for dropping evolution requirements, as there are hardly any metrics to measure it, while it's also quite hard to already define what's going to happen to the project after development. Also; it's not in the example, so where does this even come from? Evolution could be handled in a seperate chapter, but that should be more about ideas, rather than about requirements.

%!TEX root = ../report.tex
\section{Risk assessment}

<<<<<<< HEAD
=======
\begin{itemize}
	\item \req{ri}: The warning system does not detect floods in time
	\item \req{ri}: The system sends warnings of a non-excising flood (false positive), making people more negligent to future messages.
	\item \req{ri}: The system can't send messages to the necessary people because the communication platform is also destroyed by the flood.
	\item \req{ri}: The system sends incorrect information, causing extra damage.
\end{itemize}
>>>>>>> 89462d9aed40ba89fa89bfb24af73d5802479848

