%!TEX root = ../report.tex
\section{System context}
The system context is a fundamental artifact in the software architecture of a system. Developing the system context view is important, because this view is used as a mechanism to trace back to the business context, and downstream to the functional and operational architecture.

\subsection{Diagram}
The system context diagram of the system is outlined in \autoref{fig:system-context-diagram}.

\begin{figure}[H]
\centering
\includegraphics[keepaspectratio=true,width=0.7\textwidth]{Diagrams/Model/Model.jpg}
\caption{System context diagram}
\label{fig:system-context-diagram}
\end{figure}

\subsection{Users and Roles}
\begin{description}
	\item[Government] Government is the main actor of this system. Government will be notified when the system detects imminent flood. The system will also tell some important and useful information such as: location of the flood, when the flood is coming, severity of the flood, which area will get the flood, and potential citizens that will be affected by the flood. The system will communicate with the government through certain API communications in the Internet.
	\item[Emergency Services] This user will also play an important role in case of floods. Emergency services will also be notified in case of imminent flood --- the government will notify them. Emergency services will also save the people and will be present in the affected areas.
	\item[Citizens] In this system, citizens can be categorized into two group. First, citizens who do not subscribe to the service offered by the system. Second, citizens who subscribe the service. Indeed, Government will notify every citizens in affected areas. However, with subscription citizens can obtain more informations regarding the flood and how to be safe.
\end{description} 

\subsection{External Systems}
\begin{description}
	\item[Weather Forecast Provider] The \ProjectName{} will also utilize weather forecast services from third party sources. This will help the system to predict upcoming flood. The system will also use several weather forecast provider to make sure that the weather information retrieved is reliable.
	\item[Sensors] To monitor the conditions of waterways in the Netherlands, this system will gather all information about temperature, water pressure, and shifting. Together with weather forecast information collected from third party sources those information will be analyzed to predict upcoming flood.
	\item[Data Visualization] The \ProjectName{} will have an API that is accessible by other developers. By using information provided from the API, they are able to visualize data to make it more usable to broader users. 
	\item[Citizens Mobile Device] This system will communicate with subscribed citizens with their mobile devices. Government will also notify the citizens through mobile communication along with other information sources such as radio, television, and siren.
	\item[Governmental API] This system will communicate to the government in case of imminent flood through API communications in the Internet. The information then will be forwarded to responsible stakeholders in the government.
\end{description}

\subsection{Channels and Information Flows}
\begin{table}[!htbp]
	\centering
    \begin{tabular}{L{\tw{0.2}} L{\tw{0.4}}}
    \toprule
    \multicolumn{2}{c}{$Sensors \Leftrightarrow Data Collecting System$} \\ \midrule
    \textbf{Description} & The \ProjectName{} gets temperature, water pressure, and shifting information from sensors in dykes. \\
    \textbf{Connection} & Wireless, Internet \\
    \textbf{Protocol} & 4G \\
    \textbf{Data Volume} & Real time \\
    \bottomrule
    \end{tabular}
\end{table}

\begin{table}[!htbp]
	\centering
    \begin{tabular}{L{\tw{0.2}} L{\tw{0.4}}}
    \toprule
    \multicolumn{2}{c}{$Weather Forecast Provider \Leftrightarrow Data Collecting System$} \\ \midrule
    \textbf{Description} & The \ProjectName{} gets weather information from weather provider \\
    \textbf{Connection} & Wireless, Internet \\
    \textbf{Protocol} & TCP/IP \\
    \textbf{Data Volume} & Real time \\
    \bottomrule
    \end{tabular}
\end{table}

\subsection{Alternatives}

\subsubsection*{Wired Connections}
Sensors will send real time information about the dykes and waterways to the Sensor monitoring block. The system will mostly use wireless connection to push the information gathered by sensors. Wired connection can also be used to send the information. However, this option needs physical connection between sensor monitoring and the sensors itself which will make the production cost way bigger since there will be many sensors installed. Thus, the wireless connection is the most feasible method.
