%!TEX root = ../report.tex

%Citizen subscribes to service
%Sensor data is retrieved FR-1, FR-3, FR-5
%Weather data is retrieved FR-7
%Predict floud probability FR-8
%Get geo data FR-9
%The system should predict water levels FR-11, FR-12
%Monitor people should have access to a control panel FR-21

\clearpage


\section{Stories and use-cases}
This section will give an overview of the different use-cases. Figure \ref{fig:usecase-diagram} displays the use-case diagram. The diagram is made using the UML notation. In this diagram, the «after» notation is introduced to indicate that one use-case occurs after another use case in time. This provides an overview of the use-cases with their actors. The architectural important use-cases are explained in more detail in the appendix.

\begin{figure}[H]
	\centering
	\includegraphics[scale=0.5]{images/usecaseDiagramNew.png}
	\caption{Use-case diagram}
	\label{fig:usecase-diagram}
\end{figure}


%slides say:
%	Name & \\
%	Number & \\
%	Primary actor &  \\
%	Scope &  \\
%	Level &  \\
%	Extensions &  \\
%	Sub-variations & \\
