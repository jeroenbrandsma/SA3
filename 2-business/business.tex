%!TEX root = ../report.tex
\chapter{Architectural business information}
\label{ch:business}
The following section describes the different aspects of the business environment of the Smart Flood Monitor. First we will explain our vision and why there is place for us at the market. After this the product and its customers will be explained. This chapter is completed with a more detailed look at the business model and some models about the market and the financial prospect.

%Start book template
% from "Software Requirements" | (Alterded by Joris) 
% ignored the section "Background" since this has allready been discussed in the previous context chapter.

\section{Business opportunity}
There are many natural disasters happening each year all over the world. Each year these disasters take lives, waste a lot of property and money and cause social disturbance. It is expected that natural disasters will cause \$300 billion in losses annually in the upcoming decade. Climate change causes the natural disasters to get worse every year. Also the number of natural disasters has increased significantly since 1970.\\
Looking at, for example, the Indian ocean's tsunami in 2004, it looks that the damage could have been significantly reduced if the necessary people were warned. A system that would reduce the damage of natural disaster, could result in billions. \\
A system that warns and guides the people before or during the floods can result in billions being saved and will save allot of lives.

%http://www.reuters.com/article/2015/03/10/us-disaster-risk-food-idUSKBN0M61G620150310
%http://www.emdat.be/natural-disasters-trends

\section{Business objectives}
Main problem: natural disasters cause too much damage\\
Main objective: reduce natural disaster damage costs by 20\%

\begin{itemize}
	\item \req{bp}: People don't know when a natural disaster is about to happen
	\item \req{bp}: People don't know how to prepare for a natural disaster
	\item \req{bp}: People don't know what to do during a natural disaster
\end{itemize}

\begin{itemize}
	\item \req{bo}: Warn the people in and around the area of a flood, at least 1 hour in advance.
	\item \req{bo}: Inform people about the details of the flood so the right preparations are made. 
	\item \req{bo}: Inform people how to save themselves, others or goods.
\end{itemize}

\section{Success metrics}
\begin{itemize}
	\item \req{sm}: 80\% of the people who receive a warning successfully get to a safe environment in time.
	\item \req{sm}: 80\% of the people that received information (before or during) a flood, reported it helped them safe extra lives or goods.
\end{itemize}

\section{Vision statement}
The Smart Flood Monitor will cause a revolutionary innovation on the environmental monitoring market. \CompanyName will offer a system which can detect floods early and correctly. This system helps us to enforce our vision, to limit the social and financial consequences of floods and avoid the loss of human lives. 

\CompanyName will not be the only competitor on the environmental monitoring market. This means \CompanyName will have to take their own strength and weaknesses into account. If we combine these qualities with an analysis of the opportunities and weaknesses on the market, we should be able to become a core part of the future environmental monitoring systems.\\\\

Such an analysis is called a SWOT-analysis. 
\missingfigure{Create table for SWOT elements}\\
Strengths: Quality of the product and affordable price.\\ % Gerrit: Low selling price
Weaknesses: [No experience?]\\
Opportunities: Due to climate change, the market will grow. \\
Threats: New competitors will enter the market.\\\\

% Gerrit commented: Not sure if we want this? We want also low product price and profit from updates and maintenance contracts
Our unique selling point is to provide the product with best quality, combined with the best service that is available.

\section{Business risks}
\begin{itemize}
	\item \req{ri}: The warning system does not detect floods in time
	\item \req{ri}: The system sends warnings of a non-excising flood (false positive), making people more negligent to future messages.
	\item \req{ri}: The system can't send messages to the necessary people because the communication platform is also destroyed by the flood.
	\item \req{ri}: The system sends incorrect information, causing extra damage.
\end{itemize}

\section{Business assumptions and dependencies}
\begin{itemize}
	\item \req{de}: The system uses the emergency services already in place by the government in order to message the citizens.
	
\end{itemize}

\section{Scope and limitations}
\subsection{Major features / key drivers}
\begin{itemize}
	\item \req{fe}: Detect floods accurately
	\item \req{fe}: Predict floods using weather forecasts
	\item \req{fe}: The system can communicate with all necessary people.
	\item \req{fe}: The system correctly sends the right warnings and messages.
		
\end{itemize}
\subsection{Scope of Initial and subsequent releases}
The initial release will only focus on floods as a natural disaster. The subsequent releases will further increase the different kinds of natural disasters that are supported by the system.\\

\todo{Create tables and diagrams of this?}
	\begin{itemize}
		\item More individual guidance
		\item Interaction with the system, users can give input
		\item More sensor support
		\item Using multiple communication networks to send information
	\end{itemize}

%\section{Business rationale}
%\section{Product and service description}
\section{Target audience}
% Gerrit:
The product will be sold to governmental institutions.

%\section{Business model}
%\section{Roadmaps}
\section{Financial model}
% Gerrit: 
The financial model will be a low product cost. This in order to price the product low in the market. A service description for maintenance will be offered. Also updates will be sold to the customer

%\section{Competitors}



