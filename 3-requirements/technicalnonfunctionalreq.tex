%!TEX root = ../report.tex
\section{Technical non-functional requirements}
<<<<<<< HEAD

=======
>>>>>>> 89462d9aed40ba89fa89bfb24af73d5802479848
In this section, the technical non-functional requirements important to this system are discussed.

\subsection{Reliability or Resilience ? }

The system will have many connected sensors, which can have failures. \\
The system should be able to recognize such failures timely \\
The system recover from failures without the QoS or the functionality of the system being affected. \\
The margin of error tolerated has to be as low as possible .\\
The system should be able to continue functioning with the same QoS in a situation where up to 5\% of the sensors suffer from failures. \\ % WM: is 5% realistic?

\subsection{Performance}
%Speed of operation
The response time and data transmission time are about 1 sec ( ? How to calculate ) \\

\subsection{Reliability}
The response time should be about NUMBER mS
Failure rate ?
% Data transmission

\subsection{Interoperability}
%The system has dependencies on third-party systems. For example, to make predictions about the development of waterlevel, 
The system will retrieve information from water forecasting services.\\ 
The system will interact with the governement, emergency services and the citizens.\\

<<<<<<< HEAD
%Not only for input, but also for output, the system will need to interoperate with third-party systems. If the system has registered a risk of a flood, it should interact with systems of emergency services and other authorities to alert them and sent relevant information.
% TODO: how to measure? 
=======
Not only for input, but also for output, the system will need to interoperate with third-party systems. If the system has registered a risk of a flood, it should interact with systems of emergency services and other authorities to alert them and sent relevant information.

\subsection{Security}
The access to the system and datas must be controled by : a firewall, passwords,hashcodes.
The sensors are protected (connected to each others, redundancy)
% TODO: how to measure?

\subsection{Scalability}
The system will be developed with more sensors and in more places.
%New functionalities, for example GPS
>>>>>>> ed4a6bd9a7bd63aac3f0528076255e9713213e07

\subsection{Security}
The access to the system and datas is controled by a firewall and passwords .\\
Only the developers and ??? can modify datas of the system.\\
The sensors are connected to each other and control each other (some kind of redundancy).\\
All system data must be backed up every 24 hours and the backup copies stored in a secure location which is not in the same building as the system.\\
All communications between the system’s data server and clients must be encrypted.\\

\subsection{Scalability}
The system will be able to cover more areas and use more sensors in the future.

% ============================================================
% From Gerrit:
% The three main requirements:
% 1: Monitoring activities: the system should monitor activities and properties of
% rivers, waterways, dykes, such as the water level and pressure or the consistency
% of the dyke. Monitoring can be performed through different devices, e.g. analog
% and digital sensors, Unmanned Aerial Vehicles (UAVs) and Vehicular Ad-hoc
% Networks (VANETs) etc.
% 2: Warning: in case of an imminent flood, the system should issue warnings to the
% authorities and emergency services, but also directly to citizens who are
% subscribed for such messages (e.g. through SMS or mobile apps).
% 3: Guidance: the system should provide runtime information to guide its
% constituents or third parties, .e.g. the UAVs may provide information to
% VANETs embedded in vehicles crossing the area, thereby guiding vehicles
% driving towards a flood area to avoid certain routes. Meanwhile, the VANETs
% can also provide an alternative route to emergency services or citizens to avoid
% the flood area. 

% ToDo:
% Stakeholders
% Use-cases
% Make them smart
% Assign value (must/should/...
% ============================================================
