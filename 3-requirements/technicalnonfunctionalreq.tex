%!TEX root = ../report.tex
\section{Technical non-functional requirements}
This section will describe the technical aspects that are important to the system as requirements. These requirements will determine various APIs and programms that the system will rely on.

% In this section, the technical non-functional requirements important to this system are discussed.

\subsection{Reliability}
% The response time should be about NUMBER mS
% Failure rate ?
% Data transmission
\begin{longtable}{L{0.1\textwidth} L{0.12\textwidth} L{0.7\textwidth}}
	\textbf{Nr.} & \textbf{Prio}  & \textbf{Description} \\
	\reqRow{rel}{1}{Must}{Sensorsites are equiped with atleast two sensors}
	\reqRow{rel}{2}{Must}{Data from the sensors is sent via a TCP connection} % Already mention REST here?
	\bottomrule
\end{longtable}

\subsection{Resilience}
The system needs to be resilliant to recover from errors and mistakes without impacting the systems functionality.
\begin{longtable}{L{0.1\textwidth} L{0.12\textwidth} L{0.7\textwidth}}
	\textbf{Nr.} & \textbf{Prio}  & \textbf{Description} \\
	\reqRow{res}{1}{Must}{The system continues to function with the same \qos in a situation where up to 5\% of the sensors suffer from failures.} % WM: is 5% realistic?
	\reqRow{res}{2}{Must}{The system recognizes failures within an hour and recovers from them without the \qos or the functionality of the system being affected.}
	\bottomrule
\end{longtable}

\subsection{Interoperability}
% The system has dependencies on third-party systems. For example, to make predictions about the development of waterlevel, the system will need to retrieve information from water forecasting services. Not only for input, but also for output, the system will need to interoperate with third-party systems. If the system has registered a risk of a flood, it should interact with systems of emergency services and other authorities to alert them and sent relevant information.

\begin{longtable}{L{0.1\textwidth} L{0.12\textwidth} L{0.7\textwidth}}
	\textbf{Nr.} & \textbf{Prio}  & \textbf{Description} \\
	\reqRow{intr}{1}{Must}{The system pulls weather forecasts from five weather forecasting services.}
	\reqRow{intr}{2}{Must}{The system pulls water forecasts from one water forecasting service.}
	\reqRow{intr}{3}{Must}{When the system detects a high risk of flood, it warns the government automatically using the government provided API.}
	\reqRow{intr}{4}{Must}{When the system detects a high risk of flood, it warns emergency services automatically using the API provided by the emergency services.}
	\bottomrule
\end{longtable}

\subsection{Security}
\begin{longtable}{L{0.1\textwidth} L{0.12\textwidth} L{0.7\textwidth}}
	\textbf{Nr.} & \textbf{Prio}  & \textbf{Description} \\
	\reqRow{sec}{1}{Must}{Access to the system is restricted to user accounts that are stored in a database.}
	\reqRow{sec}{2}{Must}{User account information are hashed using bcrypt after being salted with 128 randomly generated characters.}
	\reqRow{sec}{3}{Must}{The system is protected by a firewall that atleast scans at the application layer, while also scanning for and preventing DDoS attacks.}
	\reqRow{sec}{4}{Must}{The system communicates with it's sensors via a REST API that only allows for HTTPS connection.} % Do we want to communicate with the sensors via REST?
	\bottomrule
\end{longtable}
% TODO: how to measure?

\subsection{Scalability}
\begin{longtable}{L{0.1\textwidth} L{0.12\textwidth} L{0.7\textwidth}}
	\textbf{Nr.} & \textbf{Prio}  & \textbf{Description} \\
	\reqRow{scale}{1}{Must}{The database and services run in parallel on a private cloud that is hosted within the Netherlands} % do we want a private cloud?\
	\reqRow{scale}{2}{Must}{Cassandra is used as database for the storage of the sensordata} % do we want to use cassandra?
	\bottomrule
\end{longtable}
% The system will be developed with more sensors and in more places.
%New functionalities, for example GPS

% ============================================================
% From Gerrit:
% The three main requirements:
% 1: Monitoring activities: the system should monitor activities and properties of
% rivers, waterways, dykes, such as the water level and pressure or the consistency
% of the dyke. Monitoring can be performed through different devices, e.g. analog
% and digital sensors, Unmanned Aerial Vehicles (UAVs) and Vehicular Ad-hoc
% Networks (VANETs) etc.
% 2: Warning: in case of an imminent flood, the system should issue warnings to the
% authorities and emergency services, but also directly to citizens who are
% subscribed for such messages (e.g. through SMS or mobile apps).
% 3: Guidance: the system should provide runtime information to guide its
% constituents or third parties, .e.g. the UAVs may provide information to
% VANETs embedded in vehicles crossing the area, thereby guiding vehicles
% driving towards a flood area to avoid certain routes. Meanwhile, the VANETs
% can also provide an alternative route to emergency services or citizens to avoid
% the flood area. 

% ToDo:
% Stakeholders
% Use-cases
% Make them smart
% Assign value (must/should/...
% ============================================================
