%!TEX root = report.tex
\setlist[itemize]{topsep=0pt,itemsep=-1ex,partopsep=1ex,parsep=1ex,after=\vspace{\baselineskip}}
\setlist[enumerate]{topsep=0pt,itemsep=-1ex,partopsep=1ex,parsep=1ex,after=\vspace{\baselineskip}}
\titleformat{\chapter}{\normalfont\LARGE\bfseries}{\thechapter}{1em}{}
\titlespacing*{\chapter}{0pt}{3.5ex plus 1ex minus .2ex}{2.3ex plus .2ex}
\newdateformat{mydate}{\twodigit{\THEDAY}{ }\shortmonthname[\THEMONTH], \THEYEAR}

% The example architecture document numbers subsubsections (3.2.1)
% And it those subsubsections are also included in the table of content
% The following commands make sure that this numbering/listing happens as in the example.

%To number subsubsections:
\setcounter{secnumdepth}{3}

%To include subsubsections in the table of contents:
\setcounter{tocdepth}{3}

%Nice custom column type
\newcolumntype{L}[1]{>{\raggedright\let\newline\\\arraybackslash\hspace{0pt}}p{#1}}

\pgfplotstableset{
	string type,
	col sep=&,
	row sep=\\,
	%every table/.append style={outfile={#1.out}}
	%generate an outfile name for every table
}

\makeatletter

\pgfplotstableset{%
		bold/.append style={
	        postproc cell content/.append code={
	                \pgfkeysalso{@cell content=\textbf{##1}}%
	        },
    	},
		UCTable/.append style={
			begin table=\begin{longtable},
			end table=\end{longtable},
			every head row/.style={
				output empty row,
				before row={},
				after row={},			
			},
			every last row/.style={%
				after row=\bottomrule,
			},
			columns/value/.style={
				bold,
				column type=L{\tw{0.2}},
			},
			columns/description/.style={
				column type=L{\tw{0.6}},
			},
			before row=\midrule,
		},
		KeyValue/.append style={
			begin table=\begin{longtable},
			end table=\end{longtable},
			every head row/.style={
				output empty row,
				before row={},
				after row={},			
			},
			every last row/.style={%
				after row=\bottomrule,
			},
			columns/key/.style={
				bold,
				column type=L{\tw{0.3}},
			},
			columns/value/.style={
				column type=L{\tw{0.2}},
			},
			before row=\midrule,
		},		
	}