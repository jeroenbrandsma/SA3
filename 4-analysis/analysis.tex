%!TEX root = ../report.tex
\chapter{Analysis}
\label{ch:analysis}

\section{Assumptions}
There are several assumptions made about the system and it's environment:

\noindent{
	\begin{enumerate}
	  \item The safety region / government has means to alert citizens in an area to evacuate.
	  \item The safety region is informed about our system and will alert citizens if needed when our system alerts the safety region.
	  \item People who are subscribed for flood warnings/guidance have a mobile phone that can receive text messages and MMS.
	  \item Sensors can be placed in the water ways and dikes in locations where they can provide representative measurements.
	  \item Information is available to the system with regards to the population density and terrain height in different areas.
	  \item A flood will not reach further than 50 km. % see SEC-7
	  \item Supplying information about the areas affected by the flood in a resolution of 5 by 5 km (as described in \ref{fr:10}) is sufficient for the government/safety region to make decisions regarding alerting and evacuating citizens.
	  \item It is possible to predict the expected water level up to 12 hours in the future.
	  \item A weather forecasting API is available that provides precipitation and wind data.
	  \item The sensors are (in some way) connected to the system.
	\end{enumerate}
}

Decisions:
\noindent{
	\begin{itemize}
	\item Security? (What if someone tries to manipulate a sensor?)
	\item One or multiple warning systems (failover)
	\item Networking? Redundancy of cables (trunked?)
	\item When to alarm? What to do if only 1 sensor warns about a flood?
	\item Security decisions?
	\item Operating system decision
	\item Storage decision
	\item Push or pull updates? (Pushing I’m hoping)
	\item Programming language(s) for the system? Web-interface, sensor monitoring, weather forecasting system, push notification system
	\item In what ways will the system detect a flood? Sensors, UAV’s, forecasting?
	\item How to detect a flood based on the forecasting? Warn beforehand? Pay extra attention to the new upcoming flood by sending extra robots/uav's?
	\end{itemize}
}

% Template for decision table:
% We removed the status from the orignal entry, because we do not use this document to get confirmation on our decisions
% \begin{table}
% \begin{tabular}{L{0.2\textwidth} L{0.6\textwidth}}
%     \textbf{Name} 			& \textbf{Topic} \\ \toprule
%     \textbf{Decision} 		& \textbf{DEC-}\textbf{\nextNrRef{dn}} \\ \midrule \midrule
%     \textbf{Problem/Issue} 	& A problem description \\ \midrule
%     \textbf{Decision} 		& The method/product that has be chosen and why\\ \midrule
%     \textbf{Alternatives} 	& \textit{Other option}\\ 
%     							& Details about the other options\\\midrule
%    \textbf{Arguments} 		& A short discussion of the cons and pro's supported by the table below\\
%    						& 	\begin{tabular}{l|lllllll|l}
% 							& 		\rot{Reliability} & \rot{Resilience} & \rot{Performance} & \rot{Interopertability} & \rot{Security} & \rot{Scalability} & \rot{Cost} & \rot{\textbf{Score}} \\ \hline 
% 									weight 				& ? & ? & ? & ? & ? & ? & ? & ?\\ \hline
% 									Mobile broadband 	& ? & ? & ? & ? & ? & ? & ? & ?\\
% 									landline 			& ? & ? & ? & ? & ? & ? & ? & ?\\
% 									satalite 		 	& ? & ? & ? & ? & ? & ? & ? & ?\\
% 									direct lines 		& ? & ? & ? & ? & ? & ? & ? & ?\\
% 								\end{tabular} \\ \bottomrule
% \end{tabular}
% \caption{Decision -- Decision name.}
% \label{table:caption_alias}
% \end{table}

A higher score means a more favourable result. Meaning a high score for costs means a low cost for the system.

\begin{tabular}{L{0.2\textwidth} L{0.6\textwidth}}
    \textbf{Name} 			& \textbf{Linux} \\ \toprule
    \textbf{Decision} 		& \textbf{DEC-}\textbf{\nextNrRef{dn}} \\ \midrule \midrule
    \textbf{Status} 		& \textbf{New} \\ \midrule
    \textbf{Problem/Issue} 	& The warning system software for the natural disasters need a platform to work on.  \\ \midrule
    \textbf{Decision} 		&  The warning system will use Linux as a platform. Based on Unix, Linux is a free platform that has proven itself and is used by many servers. It's open source meaning that everyone can check out how it works.\\ \midrule
    \textbf{Alternatives} 	& \textit{Windows}\\
    						& Operating system is a closed platform developed by one of the biggest tech companies who provide a big development environment with it.\\
    						& \textit{OpenBSD}\\
    						& A Unix-based system that is famous for it's proactive security and runs most of the Linux applications. However, some software packages aren't certified to run on OpenBSD, but are for Linux.\\
    						\midrule
    \textbf{Arguments} 		& \\
    						& 	\begin{tabular}{l|lllllll|l}
							& 		\rot{Reliability} & \rot{Resilience} & \rot{Performance} & \rot{Interopertability} & \rot{Security} & \rot{Scalability} & \rot{Cost} & \rot{\textbf{Score}} \\ \hline
							% 							rel res perf int sec sca cost
									Weight 				& 1 & 1 & 1 & 1 & 1 & 1 & 1 & \\ \hline
									Linux 			 	& 4 & 4 & 4 & 4 & 4 & 4 & 4 & 28 \\
									Windows 			& 3 & 2 & 3 & 2 & 3 & 3 & 1 & 17 \\
									OpenBSD 		 	& 5 & 5 & 4 & 4 & 5 & 4 & 4 & 31 \\
								\end{tabular} \\
    \\ \bottomrule
\end{tabular}
\todo[inline]{We either need to adjust some weights or we need to switch to OpenBSD. Also, I might be biased.}

\begin{tabular}{L{0.2\textwidth} L{0.6\textwidth}}
    \textbf{Name} 			& \textbf{Connectivity of the sensors} \\ \toprule
    \textbf{Decision} 		& \textbf{DEC-}\textbf{\nextNrRef{dn}} \\ \midrule \midrule
    \textbf{Status} 		& \textbf{New} \\ \midrule
    \textbf{Problem/Issue} 	& The sensors need to deliver their data to the system and are located outdoors with at least 100m distance between each other.  \\ \midrule
    \textbf{Decision} 		&  The sensors will send their data to the system using mobile broadband. Using cellphone towers to communicate with the system.\\ \midrule
    \textbf{Alternatives} 	& \textit{Landline}\\ 
    						& Connecting the sensors to the telephone network and use that network to communicate with the server.\\
    						\midrule
    \textbf{Arguments} 		& \\
    						& 	\begin{tabular}{l|lllllll|l}
							& 		\rot{Reliability} & \rot{Resilience} & \rot{Performance} & \rot{Interopertability} & \rot{Security} & \rot{Scalability} & \rot{Cost} & \rot{\textbf{Score}} \\ \hline 
									Weight 				& 1 & 1 & 1 & 1 & 1 & 1 & 1 & \\ \hline
									Mobile broadband 	& 4 & 4 & 4 & 4 & 2 & 5 & 4 & 27 \\
									Landline 			& 2 & 2 & 3 & 4 & 3 & 3 & 5 & 22 \\
									Satellite 		 	& 3 & 1 & 2 & 4 & 4 & 4 & 3 & 21 \\
									Direct lines 		& 2 & 1 & 5 & 2 & 5 & 1 & 1 & 17 \\
								\end{tabular} \\ \bottomrule
\end{tabular}

\begin{table}
\begin{tabular}{L{0.2\textwidth} L{0.6\textwidth}}
    \textbf{Name} 			& \textbf{Cassandra Database} \\ \toprule
    \textbf{Decision} 		& \textbf{DEC-}\textbf{\nextNrRef{dn}} \\ \midrule \midrule
    \textbf{Problem/Issue} 	& A reliable database, which is the best in scalability and availability is needed to store our data for further processing and analysis. \\ \midrule
    \textbf{Decision} 		& Smart Flood Monitoring system will use Cassandra, which will run on top of the Linux platform, to store great amount of data from huge sensor arrays needed to carry out analytics and logging.\\ \midrule
    \textbf{Alternatives} 	& \textit{Redis}\\ 
    						& Redis is a database that is best for storing data that changes rapidly with foreseeable database size which mostly fits in memory. This database is good to store real-time stock prices.\\
    						&\textit{MongoDB}\\ 
    						& MongoDB is suitable for a database that needs dynamic queries. Indexes are mainly needed to runs this database system rather than Map/Reduce functions.\\
    						&\textit{HBase}\\ 
    						& HBase is also written in Java. HBase is the database for Hadoop. This database is the best way to run Map/Reduce tasks on huge datasets.\\
   	\textbf{Arguments} 		& A short discussion of the cons and pro's is described by the table below\\
   						& 	\begin{tabular}{l|lllllll|l}
							& 		\rot{Reliability} & \rot{Resilience} & \rot{Performance} & \rot{Interoperability} & \rot{Security} & \rot{Scalability} & \rot{Cost} & \rot{\textbf{Score}} \\ \hline 
									weight 		& 1 & 1 & 1 & 1 & 1 & 1 & 4 &  \\ \hline
									Cassandra 	& 4 & 3 & 5 & 3 & 3 & 5 & 4 & 27\\
									Redis 		& 2 & 3 & 3 & 3 & 3 & 2 & 4 & 20\\
									MongoDB 	& 2 & 3 & 3 & 3 & 3 & 3 & 4 & 21\\
									HBase 		& 3 & 3 & 4 & 3 & 3 & 4 & 4 & 24\\
								\end{tabular} \\ \bottomrule
\end{tabular}
\caption{Decision -- Cassandra Database.}
\label{table:caption_alias}
\end{table}


%\begin{table}
%\begin{tabular}{L{0.2\textwidth} L{0.6\textwidth}}
% \textbf{Name} 			& \textbf{Kind of sensors} \\ \toprule
% \textbf{Decision} 		& \textbf{DEC-}\textbf{\nextNrRef{dn}} \\ \midrule \midrule
% \textbf{Problem/Issue} 	& A problem description \\ \midrule
% \textbf{Decision} 		& The method/product that has be chosen and why\\ \midrule
% \textbf{Alternatives} 	& \textit{Other option}\\ 
%    							& Details about the other options\\\midrule
%\textbf{Arguments} 		& A short discussion of the cons and pro's supported by the table below\\
%   						& 	\begin{tabular}{l|lllllll|l}
%							& 		\rot{Reliability} & \rot{Resilience} & \rot{Performance} & \rot{Interopertability} & \rot{Security} & \rot{Scalability} & \rot{Cost} & \rot{\textbf{Score}} \\ \hline 
%									weight 				& ? & ? & ? & ? & ? & ? & ? & ?\\ \hline
%									Mobile broadband 	& ? & ? & ? & ? & ? & ? & ? & ?\\
%									landline 			& ? & ? & ? & ? & ? & ? & ? & ?\\
%									satalite 		 	& ? & ? & ? & ? & ? & ? & ? & ?\\
%									direct lines 		& ? & ? & ? & ? & ? & ? & ? & ?\\
%								\end{tabular} \\ \bottomrule
%\end{tabular}
%\caption{Decision -- Decision name.}
%\label{table:caption_alias}
%\end{table}