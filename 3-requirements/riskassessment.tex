%!TEX root = ../report.tex
\section{Risk assessment}
The system is confronted by severals risks which are determined and mitigated in this section.

\begin{longtable}{L{0.1\textwidth} L{0.0\textwidth} L{0.7\textwidth}}
	\textbf{Nr.} &  & \textbf{Description} \\
	\reqRow{risk}{1}{}{The warning system does not detect floods in time}
	\reqRow{risk}{2}{}{The system sends warnings of a non-excising flood (false positive), making people more negligent to future messages.}
	\reqRow{risk}{3}{}{The system can't send messages to the necessary people because the communication platform is also destroyed by the flood.}
	\reqRow{risk}{4}{}{The system sends incorrect information, causing extra damage.}
	\reqRow{risk}{5}{}{Hacker get access to the system} % is this riskworthy?
	\reqRow{risk}{6}{}{The sensors company become bankrupt}
	\reqRow{risk}{7}{}{Competitors lowering their prices} % is this riskworthy?
	\bottomrule
\end{longtable}

<<<<<<< HEAD
The system is confronted by severals risks .\\
There are different categories of risks.


\subsection{Business}

\begin{tabular} {l | c | r }

Risk & Consequence & Solution \\
\\
Wrong estimation of the budget How to find more money? & &  \\
The money invested in the fabrication of the system isn't "covered" by the sells (shortfall/deficit).& & \\
The sensors company become bankrupt. & &\\
Reducing the price because of competition. & & \\

\end{tabular}

\subsection{Technical}
\begin{itemize}
\item \req{ri}:Hackers get access to the system \\
\item \req{ri}:Failure ( how many sensors could be damaged and the system still working ? percentage ... ) \\ 

	\item \req{ri}: The warning system does not detect floods in time
	\item \req{ri}: The system sends warnings of a non-excising flood (false positive), making people more negligent to future messages.
	%How to manage ? Someone to control ?
	\item \req{ri}: The system can't send messages to the necessary people because the communication platform is also destroyed by the flood.
	\item \req{ri}: The system sends incorrect information, causing extra damage.
\end{itemize}


\textit{(HOW TO HANDLE THOSE RISKS ? SCALE OF RISKS , PROBABILITY THAT THOSE RISKS HAPPEN , How to evaluate a risk ? Measurement ?)}



=======
% False positive GK: Is this really a risk?
% Failure ( how many sensors could be damaged and the system still working ? percentage ... ) \\ GK: This is already covered by the tnfr
% The sensors company become bankrupt \\
% Reducing the price because of competition \\

% HOW TO HANDLE THOSE RISKS ? 

% \begin{itemize}
% 	\item \req{ri}: The warning system does not detect floods in time
% 	\item \req{ri}: The system sends warnings of a non-excising flood (false positive), making people more negligent to future messages.
% 	\item \req{ri}: The system can't send messages to the necessary people because the communication platform is also destroyed by the flood.
% 	\item \req{ri}: The system sends incorrect information, causing extra damage.
% \end{itemize}

%How to evaluate a risk ? Measurement ?
>>>>>>> 7cec8d570f7d6f9ddcccd20bc3850adda846719f
