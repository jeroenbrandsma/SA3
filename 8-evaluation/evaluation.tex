%!TEX root = ../report.tex
\chapter{Architecture evaluation}
\label{ch:evaluation}
In this chapter the software architecture will be evaluated. This document started with a requirement analysis. The requirements which are defined in chapter 3 are checked if they are fulfilled in the software architecture. Another part in this chapter is the architecture evaluation using Architecture Tradeoff Analysis Method (ATAM). The architectural risks can be exposed that can harm the organization's business goals.

\section{Requirements validation}
The requirements from chapter 3 are evaluated in this section. The section is split according to chapter 3. First  the functional requirements, following the commercial requirements, and at the end the technical non-functional requirements.

\subsection{Functional requirements}
%\begin{adjustbox}{max width=\textwidth}
%\begin{longtable}[H]
\newcommand{\bo}[1]{\textbf{#1}}
	\begin{longtable}{llllL{\tw{0.1}}L{\tw{0.4}}}
		\bo{Nr.} & \bo{Priority} & \bo{Fulfilled} & \bo{Decision} & \bo{Chapter} & \bo{Remarks} \\ \toprule \endhead
		
		%		INTR-1 & Must     & Yes      &\ref{subsec:external-system} & Fulfilled by using several weather forecast API providers.       \\ \midrule \midrule
		%The system is able to receive input from water level sensors.
		\ref{fr:receive-waterlevel} 
		& Must     
		& Yes
		& \ref{hw:1}, \ref{hw:2}, \ref{hw:3} 
		& \ref{sec:hardware-overview},\ref{subsubsec:components}
		,\ref{subsec:logicalview} 
		& Fulfilled by the system providing a REST server the arduino sensor systems can use to post the sensor data to. \\ \midrule
		
		%TODO create a decision for the algorithms used?
		%The system is able to perform an analysis for the water level in the waterways based on the input from the water level sensors. 
		\ref{fr:analyze-waterlevel}  
		& Must     
		& Yes       
		& 
		&\ref{subsec:logicalview}       
		& Fulfilled by having the system use several algorithms to analyze the sensor input for abnormalities \\ \midrule
		
		%The system is able to receive input from the dike sensors.
		\ref{fr:receive-pressure}
		& Must
		& Yes
		& \ref{hw:2}
		&\ref{sec:hardware-decisions} \ref{hw:2}
		& Fulfilled by having arduino systems sent the dike sensor values over a cabled network to the central system's REST server \\ \midrule
		
		%The system is able to perform an analysis for the parameters of the dike sensors based on the input from the dike sensors.
		%TODO This has to be explained better i guess.
		\ref{fr:analyze-pressure}  
		& Must     
		& Yes        
		& \ref{hw:1}
		& \ref{subsec:logicalview}, \ref{subsec:view-process} 
		& \\ \midrule
		
		%FR-5 Must The system can store the sensor data.
\ref{fr:store-sensordata}  
& Must     
& Yes        
& \ref{dec:6}
& \ref{sec:hardware-overview}, \ref{subsec:implementview}, \ref{subsec:databaseview}
& Fulfilled by using an ElasticSearch database that is capable of storing all the data needed.\\ \midrule 

\ref{fr:retrieve-sensordata}
& Must
& Yes
& \ref{dec:6}
& \ref{subsec:implementview}, \ref{subsec:databaseview}
&Fulfilled by using an Elasticsearch database that is very robust, high available and scalable. Thereby enhancing the ability to store and receive data at all times. \\ \midrule
		
%		%The system retrieves weather forecasting data from weather forecasting services,
%which consists of predictions about the precipitation, wind data and tide information.
%This is used by the system to help in determining when a flood becomes imminent
\ref{fr:receive-weather}
& Must
& Yes
& 
&\ref{subsec:external-system}
& Fulfilled by the system contacting several API's to base the decisions on. \\ \midrule 
		
		%The system is able to detect when a flood is imminent by combining the retrieved
%sensor data and weather forecasting data.
\ref{fr:detect-flood}
& Must
& Yes
&
&\ref{subsec:logicalview}, \ref{subsubsec:proc-floodmonitor}
& Fulfilled by the system using Longitudinal Data Analysis to correlate the sensor data with the forecast data \\ \midrule 
		
%		%The system retrieves geographic information, consisting of road data, terrain height
%data and demographic data (number of civilians living in affected area) from an
%external API.
\ref{fr:receive-geographic}
& Must
& Yes
&
&\ref{subsec:external-system}
& Fulfilled by the system contacting an external GEO API over TCP/IP\\ \midrule
		
%		%The system computes the area affected by a flood, in zones of 5 by 5 km, by using the
%location data of the sensors and geographic information.
\ref{fr:compute-area}
& Must
& Yes
& \ref{dec:6}
& \ref{database-data},\ref{subsec:external-system}
& Fulfilled by the system storing the geographical data in an ElasticSearch database, capable of performing queries to get the specified area \\ \midrule 
		
%		%The system is able to perform an analysis, resulting in an estimated expected water
%level for areas which are affected by a flood, based on the water level sensor data,
%geographic data and weather forecast information.
\ref{fr:analyze-waterlevel}
& Must
& Yes
&
&\ref{sec:elaboratedmodel} \ref{subsec:logicalview}
& Fulfilled by using several different external API's for additional information together with a correlation algorithm and other analysis algorithms \\ \midrule 
		
%The system estimates how the water level in the areas affected by the flood will %develop for every hour, up to 12 hours in the future.
\ref{fr:estimate-waterlevel}
& Should
& Yes
&
& \ref{subsec:logicalview}
& Fulfilled by having the system multiple prediction algorithms to analyze the flood data\\ \midrule 

%The system can compute the number of civilians living in the areas affected by the flood.
%TODO: Still no idea what those APIs actually give. 
\ref{fr:compute-nrcivilians}
& Should
& Yes
&
& \ref{subsec:external-system}
& Fulfilled by letting the system obtain the geo information of the civilians from an external API. Then letting the system store this geo data in the ElasticSearch database, capable of executing complex geo queries. \\ \midrule

%When a flood is imminent, the system sends a warning to the safety region, containing information about the flood: the area affected by the flood, the expected water level in those areas, how the water level will develop in the coming hours and the number of civilians living in the affected area.
%TODO explain better in ch7?
\ref{fr:warn-safetyregion}
& Must
& Yes
& ~
& \ref{sec:system-context}, \ref{subsec:logicalview}, \ref{subsubsec:components}, \ref{subsec:view-process}
& Fulfilled by invoking the emergency room API with a message containing data gathered from the sensor data analysis that led to this warning.\\ \midrule 

%		\frReqRow{compute-safearea}{Must}
%		{ The system can compute a safe area, not affected by the flood, where citizens can be evacuated to in case of an (imminent) flood. }
\ref{fr:compute-safearea}
& Must
& Yes
& 
& \ref{subsec:logicalview},    \ref{subsec:external-system}
& Fulfilled by using an external geoinformation API together with Elasticsearch \\ \midrule 
		
%		\frReqRow{citizens-subscribe}{Must}
%		{ Citizens are able to subscribe to flood warnings about imminent floods. }	
\ref{fr:citizens-subscribe}
& Must
& Yes
& ~
& \ref{subsec:external-system}, \ref{subsec:logicalview}
& Fulfilled by using an external SMS service provider and providing an API that allows user registration\\ \midrule

%		\frReqRow{warn-citizens}{Must}
%		{ Citizens who are subscribed for flood warnings are warned about imminent floods by text message. }
\ref{fr:warn-citizens}
& Must
& Yes
& ~
& \ref{subsec:system-alter}
& Fulfilled by sending the warning message to the external SMS service provider,  who then distributes it\\ \midrule 

%		\frReqRow{detect-faultysensor}{Must}
%		{ The system can detect a faulty sensor, either when the sensor raises an error or when the data from the sensor is inconsistent with other sensor data. }

		%TODO add references:
		%TODO not explained good enough
\ref{fr:detect-faultysensor}
& Must
& Yes
& ~
& \ref{subsec:logicalview}
& Fulfilled by using algorithms to detect abnormalities and then verifying the abnormalities using the other sensors and algorithms\\ \midrule 

%TODO way to little in the doc about this
%		\frReqRow{controlpanel}{Must}
%		{ There is a control panel, where maintainers of the system have access to. } 
\ref{fr:controlpanel}
& Must
& Yes
& ~
& \ref{subsec:view-process}
& Fulfilled by allowing maintainers to visit a control panel site that uses the REST interface. \\ \midrule 

%TODO This control panel is not enough described (and do we still use it?)
%		\frReqRow{report-faultysensors}{Must}
%		{ The system reports faulty sensors, so they can be viewed in the control panel. }
\ref{fr:report-faultysensors}
& Must
& Yes
& ~
& \ref{subsec:view-process}
& Fulfilled by storing abnormal sensor data in a different way in the database, allowing the control panel to distinguishes the faulty sensors \\ \midrule 

%		\frReqRow{controlpanel-warnings}{Must}
%		{ Warnings of the system can be viewed in the control panel.}
\ref{fr:controlpanel-warnings}
& Must
& Yes
& ~
& \ref{subsec:view-process}
& Fulfilled by providing a control panel site the maintainers can visit that shows the warnings that the algorithms stored in the database. \\ \midrule 

%		\frReqRow{controlpanel-errors}{Must}
%		{ Errors of the system can be viewed in the control panel. }
\ref{fr:controlpanel-errors}
& Must
& ~
& ~
& \ref{subsec:view-process}
& Fulfilled by providing a control panel site the maintainers can visit that shows the errors that the algorithms stored in the database. \\ \midrule 

%		\frReqRow{controlpanel-sensors}{Must}
%		{ The readings of the sensors can be viewed in the control panel. }
\ref{fr:controlpanel-sensors}
& Must
& Yes
& ~
& \ref{subsec:view-process}
& Fulfilled by providing a control panel site that uses the REST API to get and display the sensor data.\\ \midrule 

%TODO not explained
%		Joris: The backups are kind of created by having redundancy
%		\frReqRow{make-backups}{Must}
%		{ The system can make backups of its data (configuration data etc.). }
\ref{fr:make-backups}
& Must
& Yes
& \ref{dec:7}
& 
& Fulfilled by using CentOS as an operating system. This allows the system to be backed up in various ways.\\ \midrule 

%		\frReqRow{store-backups}{Must}
%		{ The system can store created backups on a remote location.}
\ref{fr:store-backups}
& Must
& Yes
& \ref{dec:7}
& 
& Fulfilled by creating backups using rsync \\ \midrule 

%		\frReqRow{retrieve-backups}{Must}
%		{ The system can retrieve the backups it previously created.}
\ref{fr:retrieve-backups}
& Must
& Yes
& \ref{dec:7}
& ~
& Fulfilled by using rsync to backup the system to a accessible location, allowing the system to retrieve the backups\\ \midrule 

%		\frReqRow{restore-backups}{Must}
%		{ The system can restore the backups it previously created after retrieving them. }
\ref{fr:restore-backups}
& Must
& Yes
& \ref{dec:7}
& ~
& Fulfilled by letting the system rsync the backup into the current running system.\\ \midrule 

%		\frReqRow{expose-api}{Must}
%		{ The system exposes an API, allowing third parties to develop applications for guidance of the citizens during a flood. }
\ref{fr:expose-api}
& Must
& Yes
& \ref{sec:system-context}
& \ref{sec:elaboratedmodel}, \ref{subsec:logicalview}, \ref{sec:archvision}, \ref{subsec:implementview}
& Fulfilled by hosting a REST sever API\\ \midrule 

%TODO 	Can it?
%		\frReqRow{detect-extremephenomena}{Could}
%		{ The system is able to detect extreme weather phenomena, like storms etc. }
\ref{fr:detect-extremephenomena}
& Could
& Partially
& ~
& ~ 
& Partially fulfilled by gathering information from various external weather API's in combination with using a correlation analysis between the sensor values and the external weather data.\\ \midrule

%TODO explain what algorithms are used?
%		\frReqRow{uav}{Should}
%		{ The system processes and stores data collected using a UAV. }	
\ref{fr:uav}
& Should
& Yes
& \ref{hw:4}
& \ref{subsec:databaseview}
& Fulfilled by having a database capable of storing images and geo information\\ \midrule		
	\end{longtable}
%\end{longtable}
%\end{adjustbox}

	\subsection{Commercial non-functional requirements}
	\begin{table}[H]
	\begin{tabular}{lllllL{\tw{0.4}}}
	
	Nr.    & Priority & Fulfilled & Decision & Chapter & Remarks \\ \hline
	
%	\reqRow{cnfr}{Must}{The system is affordable. The initial price of the system is lower than 95\%\ of the competitors price in the same market.}
	CNFR-1 & Must     & ~        & ~ & ~         & ~       \\ \hline
	
%	\reqRow{cnfr}{Must}{The sensors have a good quality, so they do not have to be replaced often. The expected lifetime of the sensors should be at least three years.}
	CNFR-2 & Must     & ~        & ~ & ~         & ~       \\ \hline
	
	\end{tabular}
	\end{table}

	\subsection{Technical non-functional requirements}
	This subsection elaborates about technical non-functional requirements that contain reliability, availability, resilience, performance, interoperability, security, and scalability.

	\subsubsection{Reliability}
	\begin{table}[H]
	\begin{tabular}{lllllL{\tw{0.4}}}
	
	Nr.   & Priority & Fulfilled & Decision & Chapter & Remarks \\ \hline
	
%	\reqRow{rel}{Must}{Data from the sensors is sent via a TCP connection} %	
	REL-1 & Must     & ~        & ~ & ~         & ~       \\ \hline
	
%	\reqRow{rel}{Must}{The system must detect if a sensor supplies wrong measurements, which can be caused, e.g. by improper calibration or defects in the sensor.} 
	REL-2 & Must     & Yes      & ~ & ~    \ref{subsec:detecting-faulty-sensor}     & Fulfilled, faulty sensor will be detected and later be repaired.       \\ \hline
	
%	\reqRow{rel}{Must}{The system must at no time fail to detect a flood when this flood becomes imminent (\textit{false negative}).}
	REL-3 & Must     & ~        & ~ & ~         & ~       \\ \hline
	
%	\reqRow{rel}{Must}{The system must not detect a flood, when this flood is not there in reality (\textit{false positive}), on average more than once per 5 years.}
	REL-4 & Must     & ~        & ~ & ~         & ~       \\ \hline
	
	\end{tabular}
	\end{table}
	
	\subsubsection{Availability}
	\begin{table}[H]
	\begin{tabular}{lllllL{\tw{0.4}}}
	
	Nr.   & Priority & Fulfilled & Decision & Chapter & Remarks \\ \hline
	
%	\reqRow{ava}{Must}{The system must have an uptime of $99.7\%$. This effectively means, that the system should not be down for more than 2 hours per month.
	AVA-1 & Must     & Yes      & & \ref{subsec:availability}         & Fulfilled.       \\ \hline
%	\reqRow{ava}{Must}{The system must not experience a period of downtime, spanning more than 12 hours. Within twelve hours of the system going offline, it should be back up again. }
	AVA-2 & Must     & ~        & & ~         & ~       \\ \hline
% \reqRow{ava}{Must}{The system pulls weather forecasts from at least two weather forecasting services.}
	AVA-3 & Must     & ~        & & ~         & ~       \\ \hline
	
	\end{tabular}
	\end{table}
	
	\subsubsection{Resilience}
	\begin{table}[H]
	\begin{tabular}{lllllL{\tw{0.4}}}
	
	Nr.   & Priority & Fulfilled & Decision & Chapter & Remarks \\ \hline
	
%	\reqRow{res}{Must}{The system recognizes failures within half an hour}
	
	RES-1 & Must     & ~        & ~ & ~         & ~       \\ \hline
	
%	\reqRow{res}{Must}{The system recovers from failures without the \qos or the functionality of the system being affected.} % GK: split in two requirements. reduce to detect sensor failures within half an hour
	
	RES-2 & Must     & ~        & ~ & ~        & ~       \\ \hline
	
%	\reqRow{res}{Must}{All system data must be backed up every 24 hours, so that in case of data loss, this data can be restored.}
	RES-3 & Must     & Yes      & ~ & \ref{subsec:database-data}         & Fulfilled, the data is also duplicated in database.       \\ \hline
	
%	\reqRow{res}{Must}{In case of a data loss, the data should be retrieved and restored from a backup within 2 hours.}
	RES-4 & Must     & ~        & ~ & ~         & ~       \\ \hline
	
%	\reqRow{res}{Must}{Backup copies are stored in a secure location which is not in the same area as the system (50 km).}		
	RES-5 & Must     & Yes      & ~ & \ref{sec:hardware-overview}         & Fulfilled, the data centers will be located in three different places.       \\ \hline
	
	\end{tabular}
	\end{table}
	
	\subsubsection{Performance}
	\begin{table}[H]
	\begin{tabular}{lllllL{0.4\textwidth}}
	
	Nr.    & Priority & Fulfilled & Decision & Chapter & Remarks \\ \hline
	
%	\reqRow{perf}{Must}{Data is transmitted from and to the system with a minimum average speed of 10 megabits per second}
	PERF-1 & Must     & Unknown  & - & -         & Expected, not yet tested. \\ \hline
	
%	\reqRow{perf}{Must}{	The data transmission between the sensors and the system is on average at least 10 megabits per second for each sensor.}
	PERF-2 & Must     & Unknown  & - & -         & Expected, not yet tested. \\ \hline
	
%	\reqRow{perf}{Must}{	The time for the system to compute if there is a flood or not according to a critical level and the data received from the sensors is at most 5 minutes. }
	PERF-3 & Must     & Unknown  & - & -         & Expected, not yet tested. \\ \hline
	
%	\reqRow{perf}{Must}{ If an imminent flood is detected, the warning text message to citizens arrives in 5 minutes. }
	PERF-4 & Must     & Unknown  & - & -         & Expected, not yet tested. \\ \hline
	
%	\reqRow{perf}{Must}{ If an imminent flood is detected, the warning to the emergency room arrives within 1 minute. }
	PERF-5 & Must     & Unknown  & - & -         & Expected, not yet tested. \\ \hline
						
	\end{tabular}
\end{table}

\subsubsection{Interoperability}
\begin{table}[H]
	\begin{tabular}{lllllL{\tw{0.4}}}
	
		Nr.    & Priority & Fulfilled & Decision & Chapter & Remarks \\ \hline
% \reqRow{intr}{Must}{The system is able to retrieve data from different types of water level sensors and dike sensors. Also future versions of the sensors should be supported.}
		INTR-1 & Must     & Yes      & \ref{dec:3} & \ref{sec:hardware-overview} & Fulfilled by using GeoBeads and Water level sensor. \\ \hline
% \reqRow{intr}{Must}{The system is able to retrieve external data from different weather forecast, demographic and geographic APIs. Also future versions of the APIs should be supported.}
		INTR-2 & Must     & Yes      & &\ref{subsec:external-system} & Fulfilled by using several external API providers.       \\ \hline
						
	\end{tabular}
\end{table}

\subsubsection{Security}
\begin{table}[H]
	\begin{tabular}{lllllL{\tw{0.4}}}
		Nr.   & Priority & Fulfilled & Decision & Chapter & Remarks \\ \hline
% \reqRow{sec}{Must}{Access to the system is restricted to users, which are authorized and authenticated using a password protected user account.}
		SEC-1 & Must     & Yes      & & \ref{subsec:system-diagram} & Fulfilled, the user must be authorized to enter the system. \\ \hline
% \reqRow{sec}{Must}{All communication to, from and within the system is encrypted.}
		SEC-2 & Must     & Yes      & \ref{dec:9} & \ref{subsec:channels-information-flows} & Fulfilled, by using secure channel. \\ \hline
% \reqRow{sec}{Must}{User account information is stored encrypted.}
		SEC-3 & Must     & Yes      & \ref{dec:6} & \ref{subsec:database-data} & Fulfilled, applied in Elasticsearch DB. \\ \hline
% \reqRow{sec}{Must}{The system is protected on both the application layer and network layer.}
		SEC-4 & Must     & Yes      & & \ref{sec:hardware-overview} & Fulfilled, in the data center level. \\ \hline
% \reqRow{sec}{Must}{The system communicates with the sensors via a secure HTTPS connection.}
		SEC-5 & Must     & Yes      & \ref{dec:9} & \ref{sec:hardware-overview} & Fulfilled, by using Arduino to send data. \\ \hline
						
	\end{tabular}
\end{table}

\subsubsection{Scalability}
\begin{table}[H]
	\centering
	\begin{tabular}{lllllL{\tw{0.4}}}
						    
		Nr.     & Priority & Fulfilled & Decision & Chapter & Remarks \\ \hline
% \reqRow{scale}{Must}{The database and services of the system can scale within 1 hour, when the systems resource usage increases.}
		SCALE-1 & Must     & Yes      & \ref{dec:6} & \ref{subsec:database-data} & Fulfilled by adapting Elasticsearch. \\ \hline
% \reqRow{scale}{Must}{The system is configurable to run in different areas and with different sensors.}
		SCALE-2 & Must     & Yes      & & \ref{sec:hardware-overview} & Fulfilled by adapting multiple data centers and multiple sensor type. \\ \hline
% \reqRow{scale}{Must}{The system maintains the performance requirements when the number of sensors is increased and the data from sensors is expanded.}
		SCALE-3 & Must     & Unknown  & & -         & Expected, not yet tested. \\ \hline
						
	\end{tabular}
\end{table}

\clearpage
% !TEX root = ../report.tex

\begin{table}[H]
	\begin{tabular}{L{0.2\textwidth} L{0.6\textwidth}}
		\textbf{Scenario}		& \textbf{Handling faulty sensors} \\ \toprule
		\textbf{Q-Attribute(s)} & Availability, reliability \\ \midrule
		\textbf{Environment} 	& Normal operation \\ \midrule
		\textbf{Stimulus} 		& A sensor sends wrong data or stops sending data \\ \midrule
		\textbf{Response} 		& The system ignores the sensor until is has been repaired/replaced \\ \midrule
		\textbf{Design decisions} 	& \\
			\multicolumn{2}{c}{
			\begin{tabular}{l|lllll}
				\textbf{Decision} & Req. & Sensitiv. & Tradeoff & Risk & Non-Risk \\ \hline
				
				Detection algorithm & \ref{fr:detect-faultysensor}, \ref{rel:2} & \nsl{s}{faultysensor} & \nsl{to}{faultysensor} &  &   \\
				Reporting & \ref{fr:report-faultysensors} &  &  & \nsl{r}{reportfaultysensor} &  \\
				UAV & \ref{fr:uav} & \nsl{s}{uav} & \nsl{to}{uav} & \nsl{r}{uav-weather} & \\
				
			\end{tabular} 
			} \\
			\midrule
			\multicolumn{2}{c}{\compactCell{
				\textbf{Sensitivities:} 
				\begin{itemize} \setlength{\itemsep}{-15pt}
				\item \ref{s:faultysensor}: Some of the time, it will be difficult for the detection algorithm to distinguish between extreme data from a sensor caused by a fault and extreme data caused by a flood.\\
				\item \ref{s:uav}: It takes time for the UAV to be dispatched to the potential flood location.
				\end{itemize} ~\\[-0.5cm]
				\textbf{Tradeoffs:} 
				\begin{itemize} \setlength{\itemsep}{-15pt}
				\item \ref{to:faultysensor}: Reliability (+) vs. Performance (-) -- Using an algorithm on the sensor data to detect faulty sensors adds more overhead, but increases the reliability of the system.\\
				\item \ref{to:uav}: Reliability (+) vs. Performance (-) -- The UAV checks if a flood is present/developing when the sensor data is not conclusive. It has a large dispatch time, which means (if there is a flood), it will be detected with a larger delay. 
				\end{itemize} ~\\[-0.5cm]
				\textbf{Risks:} 
				\begin{itemize} \setlength{\itemsep}{-15pt}
				\item \ref{r:reportfaultysensor}: There is a risk that the config panel where the faulty sensors are reported, is not checked often enough, leading to broken sensors not being replaced.
				\end{itemize}
			}} \\

		\midrule
		\textbf{Reasoning} 		& Temporarily ignoring faulty sensors allows the system to continue functioning. Reporting the sensors using the control panel allows maintenance personnel to repair those sensors. It is important that maintenance personnel checks the control panel regularly. 
		
		In case it cannot be determined by the system if a sensor is faulty, or whether there is a flood, a UAV can be dispatched to check. \\
		%\midrule
		%\textbf{Arch. model} 	&  \\
								 
	 \bottomrule
	\end{tabular}
	\label{ATAM:faulty-sensors}
	\caption{ATAM -- Handling faulty sensors}
\end{table}

\begin{table}[H]
	\begin{tabular}{L{0.2\textwidth} L{0.6\textwidth}}
		\textbf{Scenario}		& \textbf{Handling hardware failures} \\ \toprule
		\textbf{Q-Attribute(s)} & Availability \\ \midrule
		\textbf{Environment} 	& Normal operation \\ \midrule
		\textbf{Stimulus} 		& A hardware component stops operating \\ \midrule
		\textbf{Response} 		& The system uses a backup of the hardware component \\ \midrule
		\textbf{Design decisions} 	& \\
			\multicolumn{2}{c}{
			\begin{tabular}{l|lllll}
				\textbf{Decision} & Req. & Sensitiv. & Tradeoff & Risk & Non-Risk \\ \hline
				
				Database cluster & \ref{ava:1}, \ref{ava:2} &  & \nsl{to}{cluster} &  &   \\
				Analytic cluster & \ref{ava:1}, \ref{ava:2} &  & \ref{to:cluster} &  &  \\
				Multiple data centers & \ref{ava:1}, \ref{ava:2} &  & \nsl{to}{datacentre} &  & \nsl{nr}{datacenters} \\
				Arduino               & \ref{ava:1}, \ref{ava:2} & \nsl{s}{arduinofail} &  &  &  \\
				
			\end{tabular} 
			} \\
			\midrule
			\multicolumn{2}{c}{\compactCell{
				\textbf{Sensitivities:} 
				\begin{itemize} \setlength{\itemsep}{-15pt}
				\item \ref{s:arduinofail}: The Arduino are hardware components which can fail as well. Several sensors are connected to a single Arduino. Since a failing Arduino does not have a backup, those sensors connected to it will become unavailable to the system until the Arduino is repaired. \\
				\end{itemize} ~\\[-1.0cm]
				\textbf{Tradeoffs:} 
				\begin{itemize} \setlength{\itemsep}{-15pt}
				\item \ref{to:cluster}: Availability (+) vs. Affordability (-) -- Using a cluster is more expensive, but provides a fallback in case of failures, increasing the availability. \\
				\item \ref{to:datacentre}: Also Availability (+) vs Affordability (-) -- The costs of the system increase significantly by having a second data center. However, this guarantees the availability of the system in case of problems with one of the data centers.
				\end{itemize} ~\\[-0.5cm]
				\textbf{Nonrisks:} 
				\begin{itemize} \setlength{\itemsep}{-15pt}
				\item \ref{nr:datacenters}: Multiple data centers aid with increasing the availability only when they are not placed in close proximity to each other.
				\end{itemize}
			}} \\

		\midrule
		\textbf{Reasoning} 		& The hardware in the data centers and the data center itself are prepared for failures of components. It is important to note that the data centers should not be placed in close proximity to each other. 
		
		A failure of an Arduino will lead to several sensors going offline, but these sensor are located in a relatively small area and therefore only have a limited impact on the systems monitoring capabilities. \\
		\midrule
		\textbf{Arch. model} 	& See figure~\ref{fig:database-cluster} and figure~\ref{fig:analytic-cluster} for the logical schematic of the database and analytic cluster respectively.
		
		Also see figure~\ref{fig:hardware-archi-schema} for an overview of the multiple data centers. \\
								 
	 \bottomrule
	\end{tabular}
	\caption{ATAM -- Handling hardware failures}
	\label{ATAM:hardware-failure}
\end{table}

% TODO: guarnteeing uptime of 99.xxxx %
% TODO: hardware failure


