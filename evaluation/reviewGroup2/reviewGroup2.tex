\documentclass[a4paper,10pt]{article}
\usepackage[utf8]{inputenc}
% \usepackage[dutch]{babel}
\usepackage{enumerate}
\usepackage{graphicx}
\usepackage{listings}

 
\author{Group 3}
\date{\today}
\title{Review report Group 2}

\begin{document}

\maketitle

\section{General}
Generally we found the document to be logical and well written. However, we made some comments in the document. We made the most comments on your document before the presentations, so maybe these issues are already addressed by now. In this document we will give a short summary of things that we noticed in your report. For more details we refer to the document itself, with more extended comments. 

Furthermore, the number of pages in the document overpasses the maximum number of pages allowed. It is better to make things more compact or put some detailed images to appendices. Please have a look at the Architecture Project Description to make sure that the pages meet the maximum limitation per chapter.

\section{Introduction}
You could make your target more specific. Rural and urban area of Europe are such a big area with lots of difference in terms of terrains and geographical structures. Designing an architecture that fits in this criteria is difficult thing to do.

\section{Business Information}
You mentioned in section 2.3 that your system will use SDI-12 as the protocol, which is nice to have a specific protocol to use. However, you probably can also explain it there, briefly, why do you use that protocol.

In section 2.4, it is not clear which kind of guidance you would provide. Furthermore, the target audience seems to be too broad and unclear as you mentioned that your target will be the Netherlands and abroad, but it is not declared which other countries. Are those countries your primary consideration? Or those countries will be your road map implementation?

The time range of the milestones in road map seems to be too ambitious. Are you sure that your system will be launched to three continents only in 4 years? Since it would take longer time even to implement to a more specific area. Furthermore, try to look ad section 4.3, technology roadmap. Because section 4.3 roadmap and this roadmap are not consistent.

Why did you choose Poland as a test case to calculate hardware cost? Are you planning to implement the system there? It is nice to separate the cost, but it is also better to show total cost of your system.

\section{Requirements}
Generally this chapter seems to be quite good and robust. We noticed that section 3.11 is suppose to be a subsection of section 3.10, which is evolution requirements.

You can try to make this chapter more compact. This chapter contains more that 20 pages, but in the project description, this chapter is suppose to have only 10 pages.

\section{Analysis}
\begin{itemize}
\item Decision HD2 horizontal rule bumps with the right edge of the page.

\item Some decisions do not have sufficient number of alternatives, even decision HD6 has only one alternative. It is better to have 3 alternatives.

\item The alternatives for HD4 are not well aligned so that it looks like it only has one alternative.

\item By using private cloud, as describe in HD5, can you make sure the security and privacy of sensitive data is not compromised?
\end{itemize}

\section{System Architecture}
\begin{itemize}
\item In 5.2.1, it is not motivated why a hub network is chosen over a mesh network.

\item In 5.2.3, about the alternatives for sensor power supply, the text reads like it was decided to use mains power supply, but you decided to use a hybrid, right?

\item What are the alternatives to the Alert Media component?

\item There is a cell tower in your elaborated model. Do you have access to those, is there a telecom provider as a third party? From the entire document, is unclear how the system will send warnings via cell phone towers as mentioned in use case 6.1.

\item In the elaborated model, you mention that you broadcast the warning to emergency services by mail. Is that a reliable medium and will they be read on time? Did you consider alternative ways to alert emergency services?

\item In 5.4.1 the availability is verified and estimated to be 99.56733\%. In TNR-2.7 it is stated the system is online 99.99\% for 90\% of sensors. You do not verify if 90\% of the sensors will be online all the time, i.e. if no more than 10\% of the sensors will be offline at any given moment.

\item Time to market: how about installing the hardware components, the sensors, hubs etc. It seems like this was not taken into account for the time to market.

\item Figure 10 does not make clear which connections can be both wired and wireless.


\end{itemize}

\section{Hardware Architecture}
\begin{itemize}
\item How did you get to the server specifications in 6.5?

\item Why is the processing instance 104GB (it's a very specific number) and is it not better/more affordable to have numerous smaller (in memory/cpu) instances working together?

\item In 6.6 it is stated that you need at least three instances for the databases to have optimal performance, but this is not explained.

\item Some of the decisions have the `\textit{To be reviewed}' status, but in the appendix they are approved.

\end{itemize}

\section{Software Architecture}
\begin{itemize}
\item The hubs are not in the activity diagram, but we assume the hubs do have running processes.

\item In the activity diagram, the data process does not enter a loop after the system launch (like the flood prediction process does), but instead immediately checks the source ID.

\item In the activity diagram, it looks like when the ID or data is not valid, there is no error reported.

\item Section 7.4.4 states that user can view flood info as well without logging in, but this is not reflected in the activity diagram.

\item In figure~36, the web portal calls registerUser on mongoDb. The facade-pattern mentioned before is not reflected in this figure. Figure~38 also call MongoDB directly.

\item The sequence diagram from figure~38 can use some more explanation. For example, how does the AlertProducer know the difference between the high risk and low risk users?

\item In the deployment diagram, it appears there is no software on the sensor hubs. Are there no software artifacts there for packaging and sending the data from the sensors?

\item In the decision SW5, there is a statement about Cassandra: `Cassandra is considered to have the best in-class scalability of NoSQL platforms'. Please add citations to the document for claims like these.

\end{itemize}

\section{Architecture evaluation}
\begin{itemize}
\item Some of the requirements are missing in the requirements verification, for example FR-6.2. This seems to be based on table 13, but it is unclear why some of these, like "determine the telephone stations to use for broadcasting warnings" are not architecturally relevant.

\item There are some requirements in the evaluation, where they are listed as fulfilled, but there is no remark or entry in the `traced in' column. Try to have either a remark or traced in for every requirement, so you know where it is fulfilled in the architecture.

\item In the ATAM tables, the requirements in the design decision tables are not filled in.

\item The ATAM tables uncover several risks about the architecture. Think of ways to mitigate those risks.
\end{itemize}

\section{System Evolution}
\begin{itemize}
\item It is good to see that dike sensors are in the system's evolution, since we think it is crucial to have dike sensors to predict floods. We wonder why the dike sensor are not introduced in the initial version, but did not read this in the document. Maybe a decision about this can be added.


\end{itemize}

\end{document}
