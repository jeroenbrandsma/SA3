%!TEX root = ../report.tex
\chapter{Architecture evaluation}
\label{ch:evaluation}
In this chapter the software architecture will be evaluated. This document started with a requirement analysis. The requirements which are defined in chapter 3 are checked if they are fulfilled in the software architecture. Another part in this chapter is the architecture evaluation using Architecture Tradeoff Analysis Method (ATAM). The architectural risks can be exposed that can harm the organization's business goals.

%\input{8-evalutation/requirementsverification.tex}
The requirements from chapter 3 are evaluated in this section. The section is split according to chapter 3. First  the functional requirements, following the commercial requirements, and at the end the technical non-functional requirements.

\subsection{Functional requirements}
\begin{table}[H]
	\begin{tabular}{lllll}
						    
		Nr.   & Priority & Location & Fulfilled & Remarks \\ \hline
		FR-1  & Must     & ~        & ~         & ~       \\ 
		FR-2  & Must     & ~        & ~         & ~       \\ 
		FR-3  & Must     & ~        & ~         & ~       \\ 
		FR-4  & Must     & ~        & ~         & ~       \\ 
		FR-5  & Must     & ~        & ~         & ~       \\ 
		FR-6  & Must     & ~        & ~         & ~       \\ 
		FR-7  & Must     & ~        & ~         & ~       \\ 
		FR-8  & Must     & ~        & ~         & ~       \\ 
		FR-9  & Must     & ~        & ~         & ~       \\ 
		FR-10 & Must     & ~        & ~         & ~       \\ 
		FR-11 & Must     & ~        & ~         & ~       \\ 
		FR-12 & Should   & ~        & ~         & ~       \\ 
		FR-13 & Should   & ~        & ~         & ~       \\ 
		FR-14 & Must     & ~        & ~         & ~       \\ 
		FR-15 & Must     & ~        & ~         & ~       \\ 
		FR-16 & Must     & ~        & ~         & ~       \\ 
		FR-17 & Must     & ~        & ~         & ~       \\ 
		FR-18 & Must     & ~        & ~         & ~       \\ 
		FR-19 & Must     & ~        & ~         & ~       \\ 
		FR-20 & Must     & ~        & ~         & ~       \\ 
		FR-21 & Must     & ~        & ~         & ~       \\ 
		FR-22 & Must     & ~        & ~         & ~       \\ 
		FR-23 & Must     & ~        & ~         & ~       \\ 
		FR-24 & Must     & ~        & ~         & ~       \\ 
		FR-25 & Must     & ~        & ~         & ~       \\ 
		FR-26 & Must     & ~        & ~         & ~       \\ 
		FR-27 & Must     & ~        & ~         & ~       \\ 
		FR-28 & Must     & ~        & ~         & ~       \\ 
		FR-29 & Could    & ~        & ~         & ~       \\
						
	\end{tabular}
\end{table}

\subsection{Commercial non-functional requirements}
\begin{table}[H]
	\begin{tabular}{lllll}
						    
		Nr.    & Priority & Location & Fulfilled & Remarks \\ \hline
		CNFR-1 & Must     & ~        & ~         & ~       \\ 
		CNFR-2 & Must     & ~        & ~         & ~       \\
						
	\end{tabular}
\end{table}

\subsection{Technical non-functional requirements}
\subsubsection{Reliability}
\begin{table}[H]
	\begin{tabular}{lllll}
						    
		Nr.   & Priority & Location & Fulfilled & Remarks \\ \hline
		REL-1 & Must     & ~        & ~         & ~       \\ 
		REL-2 & Must     & ~        & ~         & ~       \\ 
		REL-3 & Must     & ~        & ~         & ~       \\ 
		REL-4 & Must     & ~        & ~         & ~       \\
						
	\end{tabular}
\end{table}

\subsubsection{Availability}
\begin{table}[H]
	\begin{tabular}{lllll}
						    
		Nr.   & Priority & Location & Fulfilled & Remarks \\ \hline
		AVA-1 & Must     & ~        & ~         & ~       \\ 
		AVA-2 & Must     & ~        & ~         & ~       \\
						
	\end{tabular}
\end{table}

\subsubsection{Resilience}
\begin{table}[H]
	\begin{tabular}{lllll}
						    
		Nr.   & Priority & Location & Fulfilled & Remarks \\ \hline
		RES-1 & Must     & ~        & ~         & ~       \\ 
		RES-2 & Must     & ~        & ~         & ~       \\ 
		RES-3 & Must     & ~        & ~         & ~       \\ 
		RES-4 & Must     & ~        & ~         & ~       \\ 
		RES-5 & Must     & ~        & ~         & ~       \\
						
	\end{tabular}
\end{table}

\subsubsection{Performance}
\begin{table}[H]
	\begin{tabular}{lllll}
						    
		Nr.    & Priority & Location & Fulfilled & Remarks \\ \hline
		PERF-1 & Must     & ~        & ~         & ~       \\ 
		PERF-2 & Must     & ~        & ~         & ~       \\ 
		PERF-3 & Must     & ~        & ~         & ~       \\
						
	\end{tabular}
\end{table}

\subsubsection{Interoperability}
\begin{table}[H]
	\begin{tabular}{lllll}
						    
		Nr.    & Priority & Location & Fulfilled & Remarks \\ \hline
		INTR-1 & Must     & ~        & ~         & ~       \\ 
		INTR-2 & Must     & ~        & ~         & ~       \\ 
		INTR-3 & Must     & ~        & ~         & ~       \\ 
		INTR-4 & Must     & ~        & ~         & ~       \\ 
		INTR-5 & Must     & ~        & ~         & ~       \\ 
		INTR-6 & Must     & ~        & ~         & ~       \\ 
		INTR-7 & Must     & ~        & ~         & ~       \\
						
	\end{tabular}
\end{table}

\subsubsection{Security}
\begin{table}[H]
	\begin{tabular}{lllll}
						    
		Nr.   & Priority & Location & Fulfilled & Remarks \\ \hline
		SEC-1 & Must     & ~        & ~         & ~       \\ 
		SEC-2 & Must     & ~        & ~         & ~       \\ 
		SEC-3 & Must     & ~        & ~         & ~       \\ 
		SEC-4 & Must     & ~        & ~         & ~       \\ 
		SEC-5 & Must     & ~        & ~         & ~       \\
						
	\end{tabular}
\end{table}

\subsubsection{Scalability}
\begin{table}[H]
	\begin{tabular}{lllll}
						    
		Nr.     & Priority & Location & Fulfilled & Remarks \\ \hline
		SCALE-1 & Must     & ~        & ~         & ~       \\ 
		SCALE-2 & Must     & ~        & ~         & ~       \\ 
		SCALE-3 & Must     & ~        & ~         & ~       \\
						
	\end{tabular}
\end{table}

%% !TEX root = ../report.tex

\begin{table}[H]
	\begin{tabular}{L{0.2\textwidth} L{0.6\textwidth}}
		\textbf{Scenario}		& \textbf{Handling faulty sensors} \\ \toprule
		\textbf{Q-Attribute(s)} & Availability, reliability \\ \midrule
		\textbf{Environment} 	& Normal operation \\ \midrule
		\textbf{Stimulus} 		& A sensor sends wrong data or stops sending data \\ \midrule
		\textbf{Response} 		& The system ignores the sensor until is has been repaired/replaced \\ \midrule
		\textbf{Design decisions} 	& \\
			\multicolumn{2}{c}{
			\begin{tabular}{l|lllll}
				\textbf{Decision} & Req. & Sensitiv. & Tradeoff & Risk & Non-Risk \\ \hline
				
				Detection algorithm & \ref{fr:detect-faultysensor}, \ref{rel:2} & \nsl{s}{faultysensor} & \nsl{to}{faultysensor} &  &   \\
				Reporting & \ref{fr:report-faultysensors} &  &  & \nsl{r}{reportfaultysensor} &  \\
				UAV & \ref{fr:uav} & \nsl{s}{uav} & \nsl{to}{uav} & \nsl{r}{uav-weather} & \\
				
			\end{tabular} 
			} \\
			\midrule
			\multicolumn{2}{c}{\compactCell{
				\textbf{Sensitivities:} 
				\begin{itemize} \setlength{\itemsep}{-15pt}
				\item \ref{s:faultysensor}: Some of the time, it will be difficult for the detection algorithm to distinguish between extreme data from a sensor caused by a fault and extreme data caused by a flood.\\
				\item \ref{s:uav}: It takes time for the UAV to be dispatched to the potential flood location.
				\end{itemize} ~\\[-0.5cm]
				\textbf{Tradeoffs:} 
				\begin{itemize} \setlength{\itemsep}{-15pt}
				\item \ref{to:faultysensor}: Reliability (+) vs. Performance (-) -- Using an algorithm on the sensor data to detect faulty sensors adds more overhead, but increases the reliability of the system.\\
				\item \ref{to:uav}: Reliability (+) vs. Performance (-) -- The UAV checks if a flood is present/developing when the sensor data is not conclusive. It has a large dispatch time, which means (if there is a flood), it will be detected with a larger delay. 
				\end{itemize} ~\\[-0.5cm]
				\textbf{Risks:} 
				\begin{itemize} \setlength{\itemsep}{-15pt}
				\item \ref{r:reportfaultysensor}: There is a risk that the config panel where the faulty sensors are reported, is not checked often enough, leading to broken sensors not being replaced.
				\end{itemize}
			}} \\

		\midrule
		\textbf{Reasoning} 		& Temporarily ignoring faulty sensors allows the system to continue functioning. Reporting the sensors using the control panel allows maintenance personnel to repair those sensors. It is important that maintenance personnel checks the control panel regularly. 
		
		In case it cannot be determined by the system if a sensor is faulty, or whether there is a flood, a UAV can be dispatched to check. \\
		%\midrule
		%\textbf{Arch. model} 	&  \\
								 
	 \bottomrule
	\end{tabular}
	\label{ATAM:faulty-sensors}
	\caption{ATAM -- Handling faulty sensors}
\end{table}

\begin{table}[H]
	\begin{tabular}{L{0.2\textwidth} L{0.6\textwidth}}
		\textbf{Scenario}		& \textbf{Handling hardware failures} \\ \toprule
		\textbf{Q-Attribute(s)} & Availability \\ \midrule
		\textbf{Environment} 	& Normal operation \\ \midrule
		\textbf{Stimulus} 		& A hardware component stops operating \\ \midrule
		\textbf{Response} 		& The system uses a backup of the hardware component \\ \midrule
		\textbf{Design decisions} 	& \\
			\multicolumn{2}{c}{
			\begin{tabular}{l|lllll}
				\textbf{Decision} & Req. & Sensitiv. & Tradeoff & Risk & Non-Risk \\ \hline
				
				Database cluster & \ref{ava:1}, \ref{ava:2} &  & \nsl{to}{cluster} &  &   \\
				Analytic cluster & \ref{ava:1}, \ref{ava:2} &  & \ref{to:cluster} &  &  \\
				Multiple data centers & \ref{ava:1}, \ref{ava:2} &  & \nsl{to}{datacentre} &  & \nsl{nr}{datacenters} \\
				Arduino               & \ref{ava:1}, \ref{ava:2} & \nsl{s}{arduinofail} &  &  &  \\
				
			\end{tabular} 
			} \\
			\midrule
			\multicolumn{2}{c}{\compactCell{
				\textbf{Sensitivities:} 
				\begin{itemize} \setlength{\itemsep}{-15pt}
				\item \ref{s:arduinofail}: The Arduino are hardware components which can fail as well. Several sensors are connected to a single Arduino. Since a failing Arduino does not have a backup, those sensors connected to it will become unavailable to the system until the Arduino is repaired. \\
				\end{itemize} ~\\[-1.0cm]
				\textbf{Tradeoffs:} 
				\begin{itemize} \setlength{\itemsep}{-15pt}
				\item \ref{to:cluster}: Availability (+) vs. Affordability (-) -- Using a cluster is more expensive, but provides a fallback in case of failures, increasing the availability. \\
				\item \ref{to:datacentre}: Also Availability (+) vs Affordability (-) -- The costs of the system increase significantly by having a second data center. However, this guarantees the availability of the system in case of problems with one of the data centers.
				\end{itemize} ~\\[-0.5cm]
				\textbf{Nonrisks:} 
				\begin{itemize} \setlength{\itemsep}{-15pt}
				\item \ref{nr:datacenters}: Multiple data centers aid with increasing the availability only when they are not placed in close proximity to each other.
				\end{itemize}
			}} \\

		\midrule
		\textbf{Reasoning} 		& The hardware in the data centers and the data center itself are prepared for failures of components. It is important to note that the data centers should not be placed in close proximity to each other. 
		
		A failure of an Arduino will lead to several sensors going offline, but these sensor are located in a relatively small area and therefore only have a limited impact on the systems monitoring capabilities. \\
		\midrule
		\textbf{Arch. model} 	& See figure~\ref{fig:database-cluster} and figure~\ref{fig:analytic-cluster} for the logical schematic of the database and analytic cluster respectively.
		
		Also see figure~\ref{fig:hardware-archi-schema} for an overview of the multiple data centers. \\
								 
	 \bottomrule
	\end{tabular}
	\caption{ATAM -- Handling hardware failures}
	\label{ATAM:hardware-failure}
\end{table}

% TODO: guarnteeing uptime of 99.xxxx %
% TODO: hardware failure

The ATAM method is used to evaluate the software architecture. The purpose of ATAM is elicit and refine a precise statement of the architecture’s driving quality attribute requirements
• elicit and refine a precise statement of the architectural design decisions
• evaluate the architectural design decisions to determine if they satisfactorily address the
quality requirements [add source]
%/ source: Using the Architecture Tradeoff Analysis Method to Evaluate a Wargame Simulation System: A Case Study Lawrence G. Jones Anthony J. Lattanze December 2001 Architecture Tradeoff Analysis Initiative 

The method consists of the following steps:[add source]
1. Present the ATAM: The evaluation team presents a quick overview of the ATAM steps,
techniques used, and outputs from the process.
2. Present the business drivers: The system manager briefly presents the business drivers
and context for the architecture. 
3. Present the architecture: The architect presents an overview of the architecture.
4. Identify architectural approaches: Itemize the architectural decisions discovered in the
previous step.
5. Generate the quality attribute utility tree: Identify, prioritize, and refine the most
important quality attribute goals in a utility tree format.
6. Analyze architectural approaches: Probe the architectural approaches in light of the
quality attributes in order to identify risks, sensitivity points, and tradeoffs.
7. Brainstorm and prioritize scenarios: Create and analyze scenarios that represent the
various stakeholders’ interests to understand quality attribute requirements and their
relative importance.
8. Analyze architectural approaches: Continue to identify risks, sensitivity points, and
tradeoffs while noting the impact of each scenario on the architectural approaches.
9. Present results: Recapitulate the ATAM steps, outputs, and recommendations. 
%/ source: Using the Architecture Tradeoff Analysis Method to Evaluate a Wargame Simulation System: A Case Study Lawrence G. Jones Anthony J. Lattanze December 2001 Architecture Tradeoff Analysis Initiative 

The first three steps are presented in the previous chapters. In this sub-chapter the focus will be on step 4 and further. 

\subsection{Architectural approaches}

\subsection{Quality attribute utility tree approaches}

\subsection{Scenarios}





