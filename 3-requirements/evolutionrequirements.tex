%!TEX root = ../report.tex}
\section{Evolution requirements}
%Typical change-cases with respect to the environement, the features and the technology of the system
%http://www.agilemodeling.com/artifacts/changeCase.htm
When establishing the project, architects of the system listed a certain number of requirements which describe the features of the system. However, due to environmental changes and changing stakeholder interests, for example, the requirements may evolve.

\textbf{ER-1 : Changes in the display of the map } \\
\textit { Evolution of \ref{fr:construct-map}}. Citizens who subscribed for flood warnings get a map with specific route information (according to their location) to a safe area through the API. \\

\textbf{ER-2 : Adding of external input}\\
 Citizens can contribute to the guidance by giving extra information (for example if they identified a safe route near their location) but this information will be checked by an operator (thanks to photograph by the UAVs for example). \\
 
\textbf{ER-3 : New sensors } The system is able to work with new sensors technologies, which are coming on the market over the years. \\

\textbf{ER-4 : Improved algorithms for detecting a flood} \\
New algorithms may become available, which have a better accuracy for the flood prediction described in \ref{fr:detect-flood}. The part of the system with the flood prediction algorithm has to be modular, so that it can be replaced with an improved algorithm, once such an algorithm becomes available.

%\textbf{ER-3 : New OS } \\
%The system is able to work with other operating systems than Linux, for example Windows and MacOS. %Portability

%\textbf{ER-3} The system is able to work on several mobile devices platforms : Android, iOs, Windows Phone. \\ IN CASE OF AN APP